\chapter{Perspectivas}

Establecida la base que confirma la presencia de multifractalidad en sistemas proteicos y la posibilidad de utilizar la dimensi\'{o}n fractal como medida caracterizadora de la estructura en las prote\'{i}nas, se plantea los siguientes puntos a investigar:

\begin{enumerate}
	\item \textbf{An\'{a}lisis de la dimensi\'{o}n fractal de masa en funci\'{o}n del radio de giro ($R_g$).} 
	\item \textbf{Caracterizaci\'{o}n de la multifractalidad en sistemas proteicos exclusivamente en \'{a}tomos pesados.} La eliminaci\'{o}n de \'{a}tomos de hidr\'{o}geno permitir\'{a} aislar la contribuci\'{o}n del esqueleto proteico y el efecto de las cadenas laterales. Esto podr\'{i}a ayudar a discriminar entre la heterogeneidad estructural intr\'{i}nseca y la aportada por la distribuci\'{o}n de los \'{a}tomos ligeros, simplificando el sistema para un modelado m\'{a}s preciso.
	
	\item \textbf{Evaluaci\'{o}n de la distribuci\'{o}n fractal de \'{a}tomos ligeros y su relaci\'{o}n con la formaci\'{o}n de puentes de hidr\'{o}geno.} Un estudio centrado en la dimensi\'{o}n fractal de \'{a}tomos de hidr\'{o}geno, podr\'{i}a revelar patrones de agregaci\'{o}n espacial asociados a la formaci\'{o}n de puentes de hidr\'{o}geno. La hip\'{o}tesis nace de las regiones con una alta densidad de dichos puentes, que podr\'{i}an exhibir un valor de dimensi\'{o}n fractal de masa significativamente diferente, reflejando su papel en el plegamiento y la estabilidad de estructuras secundarias y terciarias, además de la búsqueda de una correlación entre el tipo de estructura secundaria en la proteina y el valor de $b$.
\end{enumerate}