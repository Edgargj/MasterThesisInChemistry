\chapter{Objetivos}
\section{Objetivo general}

%Determinar la posible existencia de multifractalidad en sistemas proteicos usando la dimensi\'{o}n fractal de masa que ha sido usada en sistemas magnetorreol\'{o}gicos. 

Determinar si las prote\'{i}nas presentan caracter\'{i}sticas multifractales. De ser as\'{i}, analizar dicha multifractalidad con el fin de utilizar una o m\'{a}s variantes de la
dimensi\'{o}n fractal como medida(s) caracterizadora(s)
de la estructura y patrones de agregaci\'{o}n de las prote\'{i}nas.



\subsection{Objetivos particulares}

\begin{itemize}

\item Desarrollar una herramienta de c\'{o}mputo que permita optimizar el c\'{a}lculo de la dimensi\'{o}n fractal de masa en sistemas proteicos.

\item Seleccionar un conjunto de prote\'{i}nas de estudio que tengan potencial para aplicaciones posteriores. Por ejemplo prote\'{i}nas nativas y mutadas, prote\'{i}nas que han mantenido su funci\'{o}n biol\'{o}gica pero que se han diferenciado estructuralmente, entre otros casos. Se priorizar\'{a}n los casos en que se cuente con informaci\'{o}n
experimental (\textit{PDB}).

\item Adaptar la determinaci\'{o}n de la dimensi\'{o}n fractal de masa a las prote\'{i}nas, a partir de otros casos como geles o materiales ferromagn\'{e}ticos.

\item Analizar la dimension fractal de masa de un conjunto de prote\'{i}nas, partiendo de la definici\'{o}n de multifractalidad y usando los datos experimentales recabados.

\item Analizar si la multifractalidad de una prote\'{i}na se observa con otras formas de determinar la dimensi\'{o}n fractal de masa, por ejemplo el radio de giro.

\item Explorar el uso potencial de la dimensi\'{o}n fractal de masa como una herramienta para identificar o resolver entre prote\'{i}nas gen\'{e}ticamente similares pero con diferencias estructurales notables.

%\item Determinar la posible existencia de multifractalidad en sistemas proteicos.
\end{itemize}
