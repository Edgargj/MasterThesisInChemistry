\chapter{Introducci\'{o}n}

Desde hace tiempo, se sabe que las prote\'{i}nas son pol\'{i}meros compactos. Si bien, la estructura que se captura en un cristal es representativa, en ella s\'{o}lo se observa una de las muchas formas que una prote\'{i}na puede adoptar durante el transcurso de su funci\'{o}n biol\'{o}gica. Por lo tanto, la noci\'{o}n tradicional que concibe a las prote\'{i}nas como objetos tridimensionales compactos resulta insuficiente para describir la complejidad estructural de estos sistemas macromoleculares. Es por lo anterior que, en los \'{u}ltimos años se ha señalado la posibilidad de que las prote\'{i}nas puedan estar mejor caracterizadas mediante conceptos provenientes de la geometr\'{i}a fractal en lugar de recurrir \'{u}nicamente a modelos tridimensionales convencionales \cite{Dewey1997, Mustafa1996, Vicsek1992 	, Cserzo1991}.

En este contexto, la dimensi\'{o}n fractal emerge como un par\'{a}metro relevante para cuantificar la complejidad geom\'{e}trica inherente en las estructuras proteicas. Investigaciones como las de Enright \textit{et al.} \cite{Enright2005, Enright2006} han calculado la dimensi\'{o}n fractal de varias prote\'{i}nas de forma global, sin el an\'{a}lisis de varias etapas de agregaci\'{o}n. Es decir, sin analizar c\'{o}mo cambian los patrones de agregaci\'{o}n cuando se cambia la escala de medici\'{o}n de la dimensi\'{o}n fractal de masa.


Sin embargo, las prote\'{i}nas al igual que otros sistemas complejos, presentan m\'{u}ltiples niveles de agregaci\'{o}n y conformaci\'{o}n que no se distribuyen homog\'{e}neamente en el espacio. Por lo que esta heterogeneidad podr\'{i}a sugerir la existencia de comportamientos multifractales, como se ha observado en sistemas magnetorreol\'{o}gicos (piedras, polvos magnetizados o geles envejecidos), en los que la dimensi\'{o}n fractal masa var\'{i}a a lo largo de diferentes etapas del proceso de formaci\'{o}n. Lo anterior sucede porque cuando se mide la dimensi\'{o}n fractal de masa a diferentes escalas, se puede obtener informaci\'{o}n acerca de los patrones de agregaci\'{o}n; adem\'{a}s, se ha observado la aparici\'{o}n de c\'{u}mulos  con propiedades multifractales, mostrando un comportamiento lineal. El an\'{a}lisis multifractal encontrado por Carrillo \textit{et al.} \cite{Carrillo2003}, aplicado a diferentes tipos de estructuras (no simplemente a geles envejecidos), podr\'{i}a servir para extraer informaci\'{o}n de los patrones de agregaci\'{o}n de otros objetos en diferentes etapas. 

Por otra parte, el problema de comparar diferentes estructuras de una misma prote\'{i}na contin\'{u}a siendo un desaf\'{i}o relevante en la qu\'{i}mica computacional \cite{Kufareva2012}. Una de las m\'{e}tricas m\'{a}s utilizadas para este prop\'{o}sito es la Desviaci\'{o}n Cuadr\'{a}tica Media (\textit{RMSD}, por sus siglas en ingl\'{e}s). Esta medida resulta fundamental en \'{a}reas como la bioqu\'{i}mica computacional y la modelaci\'{o}n molecular, ya que permite evaluar el grado de similitud entre estructuras obtenidas por experimentos o simulaciones moleculares. Sin embargo, el \textit{RMSD} presenta limitaciones importantes, especialmente cuando las diferencias estructurales son locales. Por ejemplo, cuando dos estructuras con diferencias en segmentos espec\'{i}ficos pueden presentar valores globales de \textit{RMSD}, ocultando variaciones funcionalmente relevantes. Por esta raz\'{o}n, es necesario desarrollar m\'{e}todos alternativos o complementarios que permitan comparar estructuras a nivel local, enfoc\'{a}ndose en regiones cr\'{i}ticas que eviten pequeñas fluctuaciones estructurales que distorsionan la comparaci\'{o}n global \cite{Kufareva2012}.\\

%Dado que las prote\'{i}nas presentan m\'{u}ltiples niveles de organizaci\'{o}n estructural y etapas de agregaci\'{o}n, surge la interrogante de si es posible detectar estos patrones estructurales mediante el an\'{a}lisis de la dimensi\'{o}n fractal de masa, bajo un esquema de objetos multifractales. De confirmarse esta hip\'{o}tesis, cabr\'{i}a preguntarse adem\'{a}s si esta magnitud es lo suficientemente sensible para distinguir entre prote\'{i}nas con secuencias gen\'{e}ticas similares pero estructuralmente diferentes. Para abordar estas preguntas, es esencial el desarrollo de una herramienta computacional que permita calcular de manera eficiente la dimensi\'{o}n fractal de masa en prote\'{i}nas, con el fin de llevar a cabo un an\'{a}lisis sistem\'{a}tico que explore el potencial de esta metodolog\'{i}a como un descriptor estructural alternativo.

Se sabe que, en las prote\'{i}nas existe m\'{a}s de una etapa de agregaci\'{o}n o de conformaci\'{o}n, esto plantea la siguiente pregunta; ¿Se podr\'{i}an detectar estos patrones de agregaci\'{o}n en las prote\'{i}nas midiendo la dimensi\'{o}n fractal de masa en este esquema de objetos multifractales? De ser cierto lo anterior, ¿ser\'{i}a esta medida lo suficientemente precisa para distinguir entre dos prote\'{i}nas con secuencias gen\'{e}ticas similares pero estructuralmente diferentes? Para responder a estas interrogantes, es fundamental desarrollar una herramienta computacional que nos permita calcular la dimensi\'{o}n fractal de masa y llevar a cabo pruebas exhaustivas. Es importante resaltar que, de ser cierto lo anterior, se podr\'{i}a abrir la puerta a futuras aplicaciones en enfermedades. No obstante, a\'{u}n no se sabe con certeza si es posible observar la multifractalidad en prote\'{i}nas, por lo que este trabajo se enfocar\'{a} en determinar si se observa.



















