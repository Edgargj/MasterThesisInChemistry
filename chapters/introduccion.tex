

\chapter{Introducci\'{o}n}
\label{chap:intro}

Las proteínas desempeñan un papel central en los sistemas biológicos, siendo indispensables para una vasta gama de funciones que sustentan la vida. Su estudio es crucial en múltiples disciplinas, desde la comprensión de procesos fisiológicos hasta el diseño de enzimas con aplicaciones biotecnológicas. La síntesis proteica, aunque no completamente elucidada, sigue un mecanismo ampliamente aceptado que implica varias etapas de plegamiento y agregación molecular. Inicialmente, los aminoácidos (unidades estructurales básicas) se organizan en estructuras secundarias localizadas ($\alpha$ hélices y laminas $\beta$). Estas, a su vez, se ensamblan en una conformación tridimensional específica (estructura terciaria), que determina la función biológica de las proteínas \cite{Hardin2022}. Finalmente, las proteínas emergen como polímeros funcionales, cuya estabilidad y actividad dependen de su correcto plegamiento. Si bien, la estructura proteica que se captura en un cristal es representativa, en ella s\'{o}lo se observa una de las muchas formas que una prote\'{i}na puede adoptar durante el transcurso de su funci\'{o}n biol\'{o}gica. Por lo tanto, la noci\'{o}n tradicional que concibe a las prote\'{i}nas como objetos tridimensionales compactos resulta insuficiente para describir la complejidad estructural de estos sistemas macromoleculares. Es por lo anterior que, en los \'{u}ltimos años se ha señalado la posibilidad de que las prote\'{i}nas puedan estar mejor caracterizadas mediante conceptos provenientes de la geometr\'{i}a fractal en lugar de recurrir \'{u}nicamente a modelos tridimensionales convencionales \cite{Dewey1997, Mustafa1996, Vicsek1992, Cserzo1991}.

En este contexto, la dimensi\'{o}n fractal emerge como un par\'{a}metro relevante para cuantificar la complejidad geom\'{e}trica inherente en las estructuras proteicas. Investigaciones como las de Enright \textit{et al.} \cite{Enright2005} han calculado la dimensi\'{o}n fractal de varias prote\'{i}nas de forma global, sin el an\'{a}lisis de varias etapas de agregaci\'{o}n. Es decir, sin analizar c\'{o}mo cambian los patrones de agregaci\'{o}n cuando se cambia la escala de medici\'{o}n de la dimensi\'{o}n fractal de masa. Otros autores como Banerji y Ghosh \cite{Banerji2011} han proporciona una revisión en las metodologías de la dimensión fractal en las proteínas, pero no reportan explícitamente el análisis multifractal en dichos sistemas. 


Sin embargo, al igual que otros sistemas complejos y como ya se dijo, las proteínas  presentan m\'{u}ltiples niveles de agregaci\'{o}n y conformaci\'{o}n que no se distribuyen homog\'{e}neamente en el espacio. Por lo que esta heterogeneidad podr\'{i}a sugerir la existencia de comportamientos multifractales, como se ha observado en sistemas magnetorreol\'{o}gicos (piedras, polvos magnetizados o geles envejecidos), en los que la dimensi\'{o}n fractal masa var\'{i}a a lo largo de diferentes etapas del proceso de formaci\'{o}n. Lo anterior sucede porque cuando se mide la dimensi\'{o}n fractal de masa a diferentes escalas, se puede obtener informaci\'{o}n acerca de los patrones de agregaci\'{o}n. El an\'{a}lisis multifractal encontrado por Carrillo \textit{et al.} \cite{Carrillo2003}, aplicado a diferentes tipos de estructuras (no simplemente a geles envejecidos), podr\'{i}a servir para extraer informaci\'{o}n de los patrones de agregaci\'{o}n de otros objetos en diferentes etapas. 

Por otra parte, el problema de comparar diferentes estructuras de una misma prote\'{i}na contin\'{u}a siendo un desaf\'{i}o relevante en la qu\'{i}mica computacional \cite{Kufareva2012}. Una de las m\'{e}tricas m\'{a}s utilizadas para este prop\'{o}sito es la desviaci\'{o}n cuadr\'{a}tica media (\textit{rmsd}, por sus siglas en ingl\'{e}s). Esta medida resulta fundamental en \'{a}reas como la bioqu\'{i}mica computacional y dinámica molecular, ya que permite evaluar el grado de similitud entre estructuras obtenidas por experimentos o simulaciones moleculares. Sin embargo, el \textit{rmsd} presenta limitaciones importantes, especialmente cuando las diferencias estructurales son locales. Por ejemplo, cuando dos estructuras con diferencias en segmentos espec\'{i}ficos  presentan valores globales de \textit{rmsd}, ocultando variaciones funcionalmente relevantes. Por esta raz\'{o}n, es necesario desarrollar m\'{e}todos alternativos o complementarios que permitan comparar estructuras a un nivel local, enfoc\'{a}ndose en regiones cr\'{i}ticas que eviten pequeñas fluctuaciones estructurales que distorsionan la comparaci\'{o}n global \cite{Kufareva2012}.

Se sabe que, en las prote\'{i}nas existe m\'{a}s de una etapa de agregaci\'{o}n o de conformaci\'{o}n, esto plantea la siguiente pregunta; ¿Se podr\'{i}an detectar estos patrones de agregaci\'{o}n en las prote\'{i}nas midiendo la dimensi\'{o}n fractal de masa en este esquema de objetos multifractales? De ser cierto lo anterior, ¿ser\'{i}a esta medida lo suficientemente precisa para distinguir entre dos prote\'{i}nas con secuencias gen\'{e}ticas similares pero estructuralmente diferentes? Para responder a estas interrogantes, es fundamental desarrollar una herramienta computacional que nos permita calcular la dimensi\'{o}n fractal de masa y llevar a cabo pruebas exhaustivas. Es importante resaltar que, de ser cierto lo anterior, se podr\'{i}a abrir la puerta a futuras aplicaciones en enfermedades. No obstante, a\'{u}n no se sabe con certeza si es posible observar la multifractalidad en prote\'{i}nas, por lo que este trabajo se enfocar\'{a} en determinar si se observa.


%Dado que las prote\'{i}nas presentan m\'{u}ltiples niveles de organizaci\'{o}n estructural y etapas de agregaci\'{o}n, surge la interrogante de si es posible detectar estos patrones estructurales mediante el an\'{a}lisis de la dimensi\'{o}n fractal de masa, bajo un esquema de objetos multifractales. De confirmarse esta hip\'{o}tesis, cabr\'{i}a preguntarse adem\'{a}s si esta magnitud es lo suficientemente sensible para distinguir entre prote\'{i}nas con secuencias gen\'{e}ticas similares pero estructuralmente diferentes. Para abordar estas preguntas, es esencial el desarrollo de una herramienta computacional que permita calcular de manera eficiente la dimensi\'{o}n fractal de masa en prote\'{i}nas, con el fin de llevar a cabo un an\'{a}lisis sistem\'{a}tico que explore el potencial de esta metodolog\'{i}a como un descriptor estructural alternativo.

\begin{comment}
	Los cristal\'{o}grafos han observado durante mucho tiempo que las prote\'{i}nas son pol\'{i}meros muy compactos. Sin embargo, la estructura nativa que se captura en un cristal es, aunque representativa, solo  una de las muchas que una prote\'{i}na puede adoptar durante el curso de su funci\'{o}n en la c\'{e}lula viva. La noci\'{o}n de que las prote\'{i}nas son simplemente objetos tridimensionales extremadamente compactos puede ser demasiado simple. De hecho, se ha señalado desde hace tiempo la posibilidad de que las prote\'{i}nas puedan estar mejor caracterizadas por la geometr\'{i}a fractal en lugar de objetos tridimensionales compactos \cite{ Dewey1997}. Adicionalmente, el problema de cuantificar las diferencia entre dos estructuras de una misma prote\'{i}na no es trivial y contin\'{u}a evolucionando \cite{Kufareva2012}. Una de las medidas  cuantitativas m\'{a}s comunes para 
	determinar la similitud entre dos conjuntos de coordenadas at\'{o}micas 
	es la Desviaci\'{o}n Cuadr\'{a}tica Media 
	(\textit{RMSD}, por sus siglas en ingl\'{e}s). El uso del \textit{RMSD} es crucial en campos como la bioqu\'{i}mica 
	y la qu\'{i}mica computacional porque permite comparar modelos generados por 
	simulaciones o experimentos para verificar su similitud con estructuras 
	conocidas, como prote\'{i}nas o complejos moleculares. 
	No obstante, el \textit{RMSD} puede ser engañoso cuando se trata de diferencias
	locales significativas. Por ejemplo, al analizar dos estructuras que difieren solo 
	en un segmento pequeño, deben tener un \textit{RMSD} global similar, a pesar de que existan partes
	altamente movibles que dificulta su an\'{a}lisis. Por esta raz\'{o}n, encontrar
	m\'{e}todos alternativos o enfoques complementarios para comparar estructuras 
	es fundamental. Estas alternativas podr\'{i}an enfocarse en regiones clave,
	permitiendo a los investigadores ignorar las variaciones menores y 
	centrarse en las partes cr\'{i}ticas de las estructuras \cite{Kufareva2012}.
	
	Por lo anterior, la dimensi\'{o}n fractal podría sintetizar toda la informaci\'{o}n que contienen los sistemas prote\'{i}cos. Por qué desde hace varios años, la dimensi\'{o}n fractal de masa se ha utilizado para identificar la distribuci\'{o}n de distintos patrones, \textit{v. gr.}, sistemas magnetoreol\'{o}gicos (piedras, polvos magnetizados o geles envejecidos) \cite{Carrillo2003}. Lo anterior sucede porque cuando se mide la dimensi\'{o}n fractal de masa a diferentes escalas, se puede obtener  informaci\'{o}n acerca de los patrones de agregaci\'{o}n; adem\'{a}s, se ha observado la aparici\'{o}n de c\'{u}mulos  con propiedades multifractales, mostrando un comportamiento lineal.
	
	Regresando a las prote\'{i}nas, autores como Matthew \textit{et al.} \cite{Enright2005} han calculado la dimensi\'{o}n fractal de masa en varias prote\'{i}nas de forma global, sin el an\'{a}lisis de varias etapas de agregaci\'{o}n. Es decir, sin analizar c\'{o}mo cambian los patrones de agregaci\'{o}n cuando se cambia la escala de medici\'{o}n de la dimensi\'{o}n fractal de masa. Otros autores como Banerji y Ghosh \cite{Banerji2011} han proporciona una revisión en las metodologías de la dimensión fractal en las proteínas, pero no reportan explícitamente el análisis multifractal en sistemas proteicos. 
	
	El an\'{a}lisis multifractal encontrado por Carrillo \textit{et al.} \cite{Carrillo2003}, aplicado a diferentes tipos de estructuras (no simplemente a geles envejecidos), podr\'{i}a servir para extraer informaci\'{o}n de los patrones de agregaci\'{o}n de otros objetos en diferentes etapas. Se sabe que, en las prote\'{i}nas existe m\'{a}s de una etapa de agregaci\'{o}n o de conformaci\'{o}n, esto plantea la siguiente pregunta; ¿Se podr\'{i}an detectar estos patrones de agregaci\'{o}n en las prote\'{i}nas midiendo la dimensi\'{o}n fractal de masa en este esquema de objetos multifractales? De ser cierto lo anterior, ¿ser\'{i}a esta medida lo suficientemente precisa para distinguir entre dos prote\'{i}nas con secuencias gen\'{e}ticas similares pero estructuralmente diferentes? Para responder a estas interrogantes, es fundamental desarrollar una herramienta computacional que nos permita calcular la dimensi\'{o}n fractal de masa y llevar a cabo pruebas exhaustivas. Es importante resaltar que, de ser cierto lo anterior, se podr\'{i}a abrir la puerta a futuras aplicaciones en enfermedades. No obstante, a\'{u}n no se sabe con certeza si es posible observar la multifractalidad en prote\'{i}nas, por lo que este trabajo se enfocar\'{a} en determinar si se observa.
\end{comment}















