\chapter{Introducción}

Desde hace tiempo, se sabe que las prote\'{i}nas son pol\'{i}meros compactos. Si bien, la estructura que se captura en un cristal es representativa, en ella s\'{o}lo se observa una de las muchas formas que una prote\'{i}na puede adoptar durante el transcurso de su funci\'{o}n biol\'{o}gica. Por lo tanto, la noci\'{o}n tradicional que concibe a las prote\'{i}nas como objetos tridimensionales compactos resulta insuficiente para describir la complejidad estructural de estos sistemas macromoleculares. Es por esto que, en los últimos años se ha señalado la posibilidad de que las prote\'{i}nas puedan estar mejor caracterizadas mediante conceptos provenientes de la geometría fractal en lugar de recurrir únicamente a modelos tridimensionales convencionales \cite{Dewey1997}.



En este contexto, la dimensión fractal emerge como un parámetro relevante para cuantificar la complejidad geométrica inherente en las estructuras proteicas. Investigaciones como las de \textit{Enright et al.}, han mostrado que la dimensión fractal de masa varía en función del tamaño de la proteína analizada, lo que sugiere que dicha magnitud puede capturar aspectos no evidentes bajo descripciones usuales. Cabe señalar que las proteínas, al igual que otros sistemas complejos, presentan múltiples niveles de agregación y conformación que no se distribuyen homogéneamente en el espacio. Esta heterogeneidad podría sugerir la existencia de comportamientos multifractales, como se ha observado en sistemas magnetorreológicos \cite{Carrillo2003}, en los que la dimensión fractal varía a lo largo de diferentes etapas del proceso de formación.\\

\clearpage

Por otra parte, el problema de comparar diferentes estructuras de una misma proteína continúa siendo un desafío relevante en la química computacional \cite{Kufareva2012}. Una de las métricas más utilizadas para este propósito es la Desviación Cuadrática Media (\textit{RMSD}, por sus siglas en inglés). Esta medida resulta fundamental en áreas como la bioquímica computacional y la modelación molecular, ya que permite evaluar el grado de similitud entre estructuras obtenidas por experimentos o simulaciones moleculares. Sin embargo, el \textit{RMSD} presenta limitaciones importantes, especialmente cuando las diferencias estructurales son locales. Por ejemplo, cuando dos estructuras con diferencias en segmentos específicos pueden presentar valores globales de \textit{RMSD}, ocultando variaciones funcionalmente relevantes. Por esta razón, es necesario desarrollar métodos alternativos o complementarios que permitan comparar estructuras a nivel local, enfocándose en regiones críticas que eviten pequeñas fluctuaciones estructurales que distorsionan la comparación global \cite{Kufareva2012}.\\

Dado que las proteínas presentan múltiples niveles de organización estructural y etapas de agregación, surge la interrogante de si es posible detectar estos patrones estructurales mediante el análisis de la dimensión fractal de masa, bajo un esquema de objetos multifractales. De confirmarse esta hipótesis, cabría preguntarse además si esta magnitud es lo suficientemente sensible para distinguir entre proteínas con secuencias genéticas similares pero estructuralmente diferentes. Para abordar estas preguntas, es esencial el desarrollo de una herramienta computacional que permita calcular de manera eficiente la dimensión fractal de masa en proteínas, con el fin de llevar a cabo un análisis sistemático que explore el potencial de esta metodología como un descriptor estructural alternativo.




