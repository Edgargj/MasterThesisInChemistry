\chapter{Introducción}

Los cristal\'{o}grafos han observado durante mucho tiempo que las prote\'{i}nas son pol\'{i}meros muy compactos. 
Sin embargo, la estructura nativa que se captura en un cristal es, aunque representativa, solo  una de las muchas 
que una prote\'{i}na puede adoptar durante el curso de su funci\'{o}n en la c\'{e}lula viva. La noci\'{o}n de que
 las prote\'{i}nas son simplemente objetos tridimensionales extremadamente compactos puede ser demasiado simple. 
 De hecho, se ha señalado desde hace tiempo la posibilidad de que las prote\'{i}nas puedan estar mejor caracterizadas 
 por la geometr\'{i}a fractal en lugar de objetos tridimensionales compactos \cite{Enright2005, Dewey1997}. Adicionalmente, 
 el problema de cuantificar las diferencia entre dos estructuras de una misma prote\'{i}na no es trivial y contin\'{u}a 
 evolucionando \cite{Kufareva2012}. Una de las medidas  cuantitativas m\'{a}s comunes para 
determinar la similitud entre dos conjuntos de coordenadas at\'{o}micas 
es la Desviaci\'{o}n Cuadr\'{a}tica Media 
(RMSD, por sus siglas en ingl\'{e}s). El uso del RMSD es crucial en campos como la bioqu\'{i}mica 
y la qu\'{i}mica computacional porque permite comparar modelos generados por 
simulaciones o experimentos para verificar su similitud con estructuras 
conocidas, como prote\'{i}nas o complejos moleculares. 
No obstante, el RMSD puede ser engañoso cuando se trata de diferencias
 locales significativas. Por ejemplo, al analizar dos estructuras que difieren solo 
 en un segmento pequeño, deben tener un RMSD global, similar a pesar de que existan partes
 altamente movibles, lo que dificulta el an\'{a}lisis. Por esta raz\'{o}n, encontrar
  m\'{e}todos alternativos o enfoques complementarios para comparar estructuras 
  es fundamental. Estas alternativas podr\'{i}an enfocarse en las regiones clave,
   permitiendo a los investigadores ignorar las variaciones menores y 
  centrarse en las partes cr\'{i}ticas de las estructuras \cite{Kufareva2012}.

Dicho lo anterior, la dimensi\'{o}n fractal puede sintetizar toda la informaci\'{o}n que contienen los sistemas prote\'{i}cos. 
Adem\'{a}s, desde hace varios años, la dimensi\'{o}n fractal de masa se ha utilizado para identificar la distribuci\'{o}n de 
distintos patrones, \textit{v. gr.}, sistemas magnetoreol\'{o}gicos (piedras, polvos magnetizados o geles envejecidos) \cite{Carrillo2003}.
 Lo anterior sucede porque cuando se mide la dimensi\'{o}n fractal de masa a diferentes escalas, se puede obtener  informaci\'{o}n acerca 
 de los patrones de agregaci\'{o}n; adem\'{a}s, se ha observado la aparici\'{o}n de c\'{u}mulos  con propiedades multifractales, mostrando 
 un comportamiento lineal.

Regresando a las prote\'{i}nas, autores como Matthew \textit{et al.} \cite{Enright2005} han calculado la dimensi\'{o}n fractal de varias 
prote\'{i}nas de forma global, sin el an\'{a}lisis de varias etapas de agregaci\'{o}n. Es decir, sin analizar c\'{o}mo cambian los patrones 
de agregaci\'{o}n cuando se cambia la escala de medici\'{o}n de la dimensi\'{o}n fractal de masa.

El an\'{a}lisis multifractal encontrado por Carrillo \textit{et al.} \cite{Carrillo2003}, aplicado a diferentes tipos de estructuras 
(no simplemente a geles envejecidos), podr\'{i}a servir para extraer informaci\'{o}n de los patrones de agregaci\'{o}n de otros objetos 
en diferentes etapas. Se sabe que, en las prote\'{i}nas existe m\'{a}s de una etapa de agregaci\'{o}n o de conformaci\'{o}n, esto plantea 
la siguiente pregunta; ¿Se podr\'{i}an detectar estos patrones de agregaci\'{o}n en las prote\'{i}nas midiendo la dimensi\'{o}n fractal de 
masa en este esquema de objetos multifractales? De ser cierto lo anterior, ¿ser\'{i}a esta medida lo suficientemente precisa para distinguir 
entre dos prote\'{i}nas con secuencias gen\'{e}ticas similares pero estructuralmente diferentes? Para responder a estas interrogantes, es 
fundamental desarrollar una herramienta computacional que nos permita calcular la dimensi\'{o}n fractal de masa y llevar a cabo pruebas 
exhaustivas. Es importante resaltar que, de ser cierto lo anterior, se podr\'{i}a abrir la puerta a futuras aplicaciones en enfermedades. 
No obstante, a\'{u}n no se sabe con certeza si es posible observar la multifractalidad en prote\'{i}nas, por lo que este trabajo 
se enfocar\'{a} en determinar si se observa.








