\chapter{Conclusi\'{o}n}

%Este trabajo establece que la multifractalidad es una propiedad de las prote\'{i}nas que integra informaci\'{o}n estructural, din\'{a}mica y funcional, proporcionando un nuevo marco cuantitativo para entender su complejidad jer\'{a}rquica. Los patrones identificados sugieren que la evoluci\'{o}n ha optimizado no solo la estructura promedio, sino tambi\'{e}n su geometr\'{i}a fractal subyacente. Por lo tanto:

\color{blue}

Este trabajo establece que las prote\'{i}nas presentan multifractalidad. Esta multifractalidad puede correlacionarse con la informaci\'{o}n din\'{a}mica y funcional de las prote\'{i}nas. Se especula que en el proceso evolutivo de las prote\'{i}nas la multifractalidad es una propiedad que proporciona un nuevo marco cuantitativo para entender su complejidad jer\'{a}rquica. En conclusi\'{o}n:


\begin{itemize}
	\item La medida del c\'{a}lculo de la dimensi\'{o}n fractal de masa demuestra sensibilidad en
	 cambios estructurales. Por consiguiente, podr\'{i}a utilizarse como medida de caracterizaci\'{o}n estructural de las prote\'{i}nas. (ver \ref{EfectoIones} y \ref{tubulinas})
	
	\item Las nueve prote\'{i}nas analizadas exhiben multifractalidad, evidenciada por la variaci\'{o}n 
	significativa en $D$ entre los intervalos 1--2~$\textup{\r{A}}$, 2--6~$\textup{\r{A}}$ y 6--20~$\textup{\r{A}}$. (ver \ref{EfectoIones})
	
	\item Estructuras cuaternarias como las prote\'{i}nas \textit{Hemoglobin} y \textit{Transformadora RHOA} podr\'{i}an arrojar una dimensi\'{o}n fractal de masa mayor a tres ($D>3$).(ver \ref{DFE})
	
	
	\item La dimensi\'{o}n fractal de masa puede cuantificar los diferentes ambientes fisicoqu\'{i}micos simulados durante una din\'{a}mica molecular, (lo cual se manifiesta en modificaciones estructurales en las conformaciones proteicas evaluadas en distintos estados del sistema). (ver apéndice \ref{9P})
	
 \color{black}
 
\end{itemize}



\chapter*{Perspectivas}

Establecida la base de los resultados que confirman la existencia de multifractalidad en sistemas proteicos y la posibilidad de utilizar la dimensión fractal como medida de caracterizadora de  la estructura en las prote\'{i}nas, se plantean las siguientes líneas de investigación:

\begin{enumerate}
	\item \textbf{Análisis de la dimensión fractal de masa en función del radio de giro ($R_g$).} 
	\item \textbf{Uso de sistemas proteicos sin la existencia de átomos de hidrógeno.}
	Solo átomos pesados.
	\item \textbf{Uso de sistemas proteicos sin la existencia de átomos de carbono.}
	Sólo átomos ligeros (gran cantidad átomos comunes). Si existen gran cantidad de puentes de hidrógeno (pudiera tener información que ciertas proteínas tienen mayor agregación de átomos de hidrógeno en regiones de cadenas proteicas con mayor dimensión fractal de dichos átomos.
\end{enumerate}

