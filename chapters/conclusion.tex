\chapter{Conclusi\'{o}n}

%Este trabajo establece que la multifractalidad es una propiedad de las prote\'{i}nas que integra informaci\'{o}n estructural, din\'{a}mica y funcional, proporcionando un nuevo marco cuantitativo para entender su complejidad jer\'{a}rquica. Los patrones identificados sugieren que la evoluci\'{o}n ha optimizado no solo la estructura promedio, sino tambi\'{e}n su geometr\'{i}a fractal subyacente. Por lo tanto:

\color{blue}

Este trabajo establece que las prote\'{i}nas presentan multifractalidad. Esta multifractalidad puede correlacionarse con la informaci\'{o}n din\'{a}mica y funcional de las prote\'{i}nas. Se especula que en el proceso evolutivo de las prote\'{i}nas la multifractalidad es una propiedad que proporciona un nuevo marco cuantitativo para entender su complejidad jer\'{a}rquica. En conclusi\'{o}n:


\begin{itemize}
	\item La medida del c\'{a}lculo de la dimensi\'{o}n fractal de masa demuestra sensibilidad en
	 cambios estructurales. Por consiguiente, podr\'{i}a utilizarse como medida de caracterizaci\'{o}n estructural de las prote\'{i}nas.
	
	\item Las nueve prote\'{i}nas analizadas exhiben multifractalidad, evidenciada por la variaci\'{o}n 
	significativa en $D$ entre los intervalos 1--2~$\textup{\r{A}}$, 2--6~$\textup{\r{A}}$ y 6--20~$\textup{\r{A}}$.
	
	\item Estructuras cuaternarias como las prote\'{i}nas \textit{Hemoglobin} y \textit{Transformadora RHOA} podr\'{i}an arrojar una dimensi\'{o}n fractal de masa mayor a tres ($D>3$).
	
	
	\item La dimensi\'{o}n fractal de masa puede cuantificar los diferentes ambientes fisicoqu\'{i}micos simulados durante una din\'{a}mica molecular, (lo cual se manifiesta en modificaciones estructurales en las conformaciones proteicas evaluadas en distintos estados del sistema).
	
 \color{black}
 
\end{itemize}
