\chapter{Conclusi\'{o}n}

%Este trabajo establece que la multifractalidad es una propiedad de las prote\'{i}nas que integra informaci\'{o}n estructural, din\'{a}mica y funcional, proporcionando un nuevo marco cuantitativo para entender su complejidad jer\'{a}rquica. Los patrones identificados sugieren que la evoluci\'{o}n ha optimizado no solo la estructura promedio, sino tambi\'{e}n su geometr\'{i}a fractal subyacente. Por lo tanto:

\color{blue}

Este trabajo establece que las prote\'{i}nas presentan multifractalidad. Esta multifractalidad puede correlacionarse con la informaci\'{o}n din\'{a}mica y funcional de las prote\'{i}nas. Se especula que en el proceso evolutivo de las prote\'{i}nas la multifractalidad es una propiedad que proporciona un nuevo marco cuantitativo para entender su complejidad jer\'{a}rquica. En conclusi\'{o}n:


\begin{itemize}
	\item La medida del c\'{a}lculo de la dimensi\'{o}n fractal de masa demuestra sensibilidad en
	 cambios estructurales. Por consiguiente, podr\'{i}a utilizarse como medida de caracterizaci\'{o}n estructural de las prote\'{i}nas. (ver \ref{subsec:EfectoIones} y \ref{subsec:tubulinas})
	
	\item Las nueve prote\'{i}nas analizadas exhiben multifractalidad, evidenciada por la variaci\'{o}n 
	significativa en $D$ entre los intervalos 1--2~$\textup{\r{A}}$, 2--6~$\textup{\r{A}}$ y 6--20~$\textup{\r{A}}$. (ver \ref{subsec:EfectoIones})
	
	\item Estructuras cuaternarias como las prote\'{i}nas \textit{Hemoglobin humana} y \textit{Hemogoblina embrionaria} podr\'{i}an arrojar una dimensi\'{o}n fractal de masa mayor a tres ($D>3$).(ver \ref{subsec:DFE})
	
	
	\item La dimensi\'{o}n fractal de masa puede cuantificar los diferentes ambientes fisicoqu\'{i}micos simulados durante una din\'{a}mica molecular, (lo cual se manifiesta en modificaciones estructurales en las conformaciones proteicas evaluadas en distintos estados del sistema). (ver \ref{subsec:DFE} y apéndice \ref{9P})
	
 \color{black}
 
\end{itemize}



\chapter*{Perspectivas}

Establecida la base de los resultados que confirman la existencia de multifractalidad en sistemas proteicos y la posibilidad de utilizar la dimensión fractal como medida de caracterizadora de  la estructura en las prote\'{i}nas, se plantean las siguientes líneas de investigación:

\begin{enumerate}
	\item \textbf{Análisis de la dimensión fractal de masa en función del radio de giro ($R_g$).} 
	\item \textbf{Caracterización de la multifractalidad en sistemas proteicos exclusivamente de átomos pesados.} La eliminación de los átomos de hidrógeno permitiría aislar la contribución del esqueleto proteico y de las cadenas laterales más masivas a la multifractalidad. Esto podría ayudar a discriminar entre la heterogeneidad estructural intrínseca y la aportada por la distribución de los átomos ligeros, simplificando el sistema para un modelado más preciso.

	\item \textbf{Evaluación de la distribución fractal de átomos ligeros y su relación con la formación de puentes de hidrógeno.} Un estudio centrado en la dimensión fractal de átomos de hidrógeno, podría revelar patrones de agregación espacial asociados a la formación de puentes de hidrógeno. La hipótesis nace de las regiones con una alta densidad de dichos puentes podrían exhibir un valor de dimensión fractal de masa para los átomos ligeros significativamente diferente, reflejando su papel en el plegamiento y la estabilidad de estructuras secundarias y terciarias.
\end{enumerate}







