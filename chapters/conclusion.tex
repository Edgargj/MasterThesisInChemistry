\chapter{Conclusión}

Este trabajo establece que la multifractalidad es una propiedad de las prote\'{i}nas que integra 
informaci\'{o}n estructural, din\'{a}mica y funcional, proporcionando un nuevo marco cuantitativo 
para entender su complejidad jer\'{a}rquica. Los patrones identificados sugieren que la evoluci\'{o}n 
ha optimizado no solo la estructura promedio, sino tambi\'{e}n su geometr\'{i}a fractal subyacente. 
Por lo tanto:

\begin{itemize}
	\item La medida del c\'{a}lculo de la dimensi\'{o}n fractal demuestra sensibilidad en
	 cambios estructurales. Por consiguiente, podr\'{i}a utilizarse como medida de caracterizaci\'{o}n 
	 estructural de las prote\'{i}nas.
	
	\item Las nueve prote\'{i}nas analizadas exhiben multifractalidad, evidenciada por la variaci\'{o}n 
	significativa en $D$ entre los intervalos 1--2~$\textup{\r{A}}$, 2--6~$\textup{\r{A}}$ y 6--20~$\textup{\r{A}}$.
	
	\item La etapa de equilibrio presenta la mayor estabilidad, con valores altos de $R^2$ y 
	dimensiones fractales consistentes, constituyendo un punto de referencia \'{o}ptimo para comparaciones estructurales.
	
	\item Estructuras cuaternarias como 1a2b, 1a3n, 1a9w podr\'{i}an arrojar una dimensi\'{o}n fractal mayor a tres ($D>3$).
	
	
	\item La din\'{a}mica molecular afecta el comportamiento de $D$ y consecuentemente, la multifractalidad de las prote\'{i}nas.
\end{itemize}
