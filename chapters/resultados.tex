\chapter{Resultados y discusión}

El propósito principal del programa es determinar la dimesi\'{o}n fractal de masa mediante la caracterizaci\'{o}n de la geometría interna de un sistema molecular. Esto se logra construyendo una tabla de \(r_i\), \( \bar{N}(r_i)\) y \( \bar{M}(r_i)\), donde;

\begin{itemize}
	\item \(r_i\) es el radio de medida centrado en posiciones aleatorias de una prote\'{i}na con valores m\'{i}nimos y m\'{a}ximos definidos como \(mr_{min}\) y \(mr_{max}\).
	\item \( \bar{N}(r_i)\) es el número promedio de partículas contenidas dentro de un radio \(r_i\). 
	\item  \( \bar{M}(r_i)\) es l n\'{u}mero de masa promedio de las part\'{i}culas contenidas en el radio \(r_i\).
\end{itemize}

Posteriormente, los valores de \( \bar{N}(r_i) \) y \( \bar{M}(r_i)\) pueden analizarse mediante ajuste logarítmico para determinar la dimensión fractal de masa del sistema. Para calcular los valores de \(r_i\), \( \bar{N}(r_i)\) y \( \bar{M}(r_i)\), el algoritmo sigue los siguientes pasos:


\begin{enumerate}
	\item Inicializaci\'{o}n\\
	 Se carga un objeto llamado Molecule que contiene la informaci\'{o}n estructural del sistema (coordenadas y masas de los elementos presentes en la prote\'{i}na).
	
	\item Configuraci\'{o}n de par\'{a}metros\\
	Se calcula el centro de masas del cl\'{u}ster.\\
	Se recentran las part\'{i}culas de la prote\'{i}na en el origen del sistema de coordenadas.\\
	Se define el radio m\'{a}ximo del cluster \(R_{max}\) (la mayor distancia entre un \'{a}tomo y el origen).\\
	Se define el radio m\'{a}ximo de b\'{u}squeda de semillas \(seekR_{max}\) (evita que los c\'{i}rculos de medici\'{o}n se salgan del cl\'{u}ster).\\
	Se definen los radios m\'{i}nimo y m\'{a}ximo de medida (\(mr_{min}\) y \(mr_{max}\)).\\
	Se calcula el n\'{u}mero de divisiones (\(nr\)) y el incremento radial (\(dr\)).\\
	Se define el n\'{u}mero de medidas por centro (nMeas) y el n\'{u}mero total de c\'{i}rculos a considerar.
	
	\item Generaci\'{o}n de puntos de medida\\
	Se eligen de manera aleatoria part\'{i}culas del sistema como centros de las circunferencias de medida. Para cada centro, se identifican las part\'{i}culas contenidas dentro del radio m\'{a}ximo (\(R_{max}\)).
	
	\item Acumulaci\'{o}n de datos\\
	Para cada radio \( r_i \), se cuenta cu\'{a}ntas part\'{i}culas est\'{a}n contenidas dentro de ese radio respecto al centro actual. Esta cuenta se repite para m\'{u}ltiples centros y se promedia para obtener \( \bar{N}(r_i) \).
	
	
	\item Salida de resultados\\
	Los vectores \(r_{i}\), \( \bar{N}(r_i) \) y \( \bar{M}(r_i)\) pueden visualizarse en pantalla o guardarse en un archivo externo.
\end{enumerate}




