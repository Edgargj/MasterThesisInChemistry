\chapter{Resultados y discusi\'{o}n}
\label{chap:resulyd}

Uno de los principales objetivos de esta tesis fue desarrollar un programa de c\'{o}mputo cient\'{i}fico para explorar la aplicaci\'{o}n del an\'{a}lisis multifractal en estructuras proteicas. Por lo tanto, el programa podr\'{i}a ser una herramienta \'{u}til para caracterizar las distintas etapas del plegamiento proteico, describi\'{e}ndolas en t\'{e}rminos de la din\'{a}mica de formaci\'{o}n de agregados estructurales, lo que permitir\'{i}a una comprensi\'{o}n m\'{a}s profunda de estos procesos. En la siguiente secci\'{o}n se describe el algoritmo del programa \textit{molmassfractraldim}. 
 
 
\section{Algoritmo (diseño del programa)}
\label{sec:algoritmo}

El prop\'{o}sito principal del programa \textit{molmassfractaldim} es determinar la dimesi\'{o}n fractal de masa mediante la caracterizaci\'{o}n de la geometr\'{i}a interna de un sistema proteico. Esto se logra construyendo una tabla de $r_{k}$, $\langle N(r_k) \rangle$, $\langle M(r_k) \rangle$ y $\langle Rg(r_k) \rangle$ donde;

\begin{itemize}
	\item $r_{k}$ es el radio de medida centrado en posiciones aleatorias de una prote\'{i}na con valores m\'{i}nimos y m\'{a}ximos definidos como $r_{min}$ y $r_{max}$.
	\item $\langle N(r_k) \rangle$ es el n\'{u}mero promedio de part\'{i}culas contenidas dentro de un radio $r_{k}$. 
	\item  $\langle M(r_k) \rangle$ es el n\'{u}mero de masa promedio de las part\'{i}culas contenidas en el radio $r_{k}$.
	\item  $\langle Rg(r_k) \rangle$ es el radio de giro promedio de las part\'{i}culas contenidas en el radio $r_{k}$.
\end{itemize}

Posteriormente, los valores de $\langle N(r_k) \rangle$, $\langle M(r_k) \rangle$ y $\langle Rg(r_k) \rangle$ pueden analizarse mediante regresiones lineales para determinar la dimensi\'{o}n fractal de masa. El algoritmo general del programa se describe en el diagrama de flujo de la figura \ref{dfMolFractalDim}, cuyos pasos se explican en las siguientes subsecciones. Para una consulta general sobre las opciones disponibles de \textit{molmassfractaldim}, v\'{e}ase el ap\'{e}ndice \ref{chap:funcmolmass}.

 \begin{figure}[h!]
 	\begin{center}
 		\includegraphics[width=\textwidth]{graphs/dfMolFractalDim}
 		\caption{Diagrama de flujo general del programa \textit{molmassfractaldim}.}
 		\label{dfMolFractalDim}
 	\end{center}
 \end{figure}
 

 \subsection {Pasos 1, 2, 3 y 4 del diagrama de flujo general}
 \label{subsec:paso1a4}
 
	\begin{itemize}
		\item \textbf{Paso 1:}
		Sea $\mathcal{M}$ un sistema molecular compuesto por un conjunto de part\'{i}culas $N_{\text{at}}$. Cada part\'{i}cula $i$ est\'{a} descrita por la terna \ref{eq:terna}.
		
		\begin{equation}
			\mathcal{M} (\mathbf{r}_i, m_i, Z_i)
			\label{eq:terna}
		\end{equation}
		
		Donde $\mathbf{r}_i$ es el vector de posici\'{o}n, $m_i$ la masa y $Z_i$ el n\'{u}mero at\'{o}mico. El sistema completo se representa por la ecuaci\'{o}n \ref{eq:sistema}.	
		
		\begin{equation}
			\mathcal{M} = \bigl\{(\mathbf{r}_i, m_i, Z_i) \,\big|\, i = 1,\dots,N_{\text{at}}\bigr\}
			\label{eq:sistema}
		\end{equation}
	
		\item \textbf{Paso 2}: Para usar las posiciones de las part\'{i}culas $i$ de $\mathcal{M}$ se define una matriz $\mathbf{X}$, que servir\'{a} como base para todos los c\'{a}lculos posteriores, v\'{e}ase la ecuaci\'{o}n \ref{eq:xmatrix}.
		\begin{equation}
			\mathbf{X} = \begin{bmatrix}
				x_1 & y_1 & z_1 \\
				x_2 & y_2 & z_2 \\
				\vdots & \vdots & \vdots \\
				x_{N_{\text{at}}} & y_{N_{\text{at}}} & z_{N_{\text{at}}}
			\end{bmatrix} \in \mathbb{R}^{N_{\text{at}} \times 3},
			\label{eq:xmatrix}
		\end{equation}
		
		Donde la $i$-\'{e}sima fila es la traspuesta $\mathbf{r}_i^{\mathsf T}$.
		
		
		\item \textbf{Paso 3}: El centro de masas $\mathbf{r}_{\text{cm}}$ del sistema $\mathcal{M}$ se calcula con la ecuaci\'{o}n \ref{eq:cm}.
		\begin{equation}
			\mathbf{r}_{\text{cm}}
			= \frac{\displaystyle \sum_{i=1}^{N_{\text{at}}} m_i \mathbf{r}_i}
			{\displaystyle \sum_{i=1}^{N_{\text{at}}} m_i}
			\in \mathbb{R}^3
			\label{eq:cm}
		\end{equation}
		$\mathbf{r}_{\text{cm}}$ representa el punto de equilibrio mec\'{a}nico del sistema $\mathcal{M}$. 
		Donde $\displaystyle \sum_{i=1}^{N_{\text{at}}} m_i \mathbf{r}_i$ es la suma de la posiciones ponderadas por las masas y $\displaystyle \sum_{i=1}^{N_{\text{at}}} m_i$ es la masa total del sistema.
		
		
		\item \textbf{Paso 4}: Se trasladan las coordenadas de $\mathcal{M}$ a su centro de masa usando las ecuaciones \ref{eq:cm2} y \ref{eq:cm3}.
		\begin{equation}
			\mathbf{r}_i' = \mathbf{r}_i - \mathbf{r}_{\text{cm}}, \qquad
			i=1,\dots,N_{\text{at}}
			\label{eq:cm2}
		\end{equation}
		
		Esto garantiza que
		\begin{equation}
			\sum_{i=1}^{N_{\text{at}}} m_i \mathbf{r}_i' = \mathbf{0}
			\label{eq:cm3}
		\end{equation} 
		
		Donde $\mathbf{r}_i'$ es la posici\'{o}n de la part\'{i}cula $i$ medida 
		con respecto al centro de masa del sistema $\mathcal{M}$ y 
		$r_{cm}$ es el vector del centro de masa del sistema $\mathcal{M}$.
		
		
	\end{itemize}


	\subsection{Paso 5 del diagrama de flujo general}
	 \label{subsec:paso5}
	
	\begin{itemize}
		
		\item \textbf{Radio m\'{a}ximo}. Se define el radio m\'{a}ximo del sistema $\mathcal{M}$, calculado como la distancia m\'{a}xima desde el origen hasta la part\'{i}cula m\'{a}s lejana del sistema, vease la ecuaci\'{o}n \ref{eq:Rmax}.
		
		\begin{equation}
			R_{\text{max}} = \max_i \|\mathbf{r}_i'\|
			\label{eq:Rmax}
		\end{equation}
		
		
		\item \textbf{Radio de b\'{u}squeda}. Se establece un radio que asegure que los puntos de medici\'{o}n no se encuentren cerca de los l\'{i}mites del sistema (ecuaci\'{o}n \ref{eq:Rseek}), con un valor del 75\% tamaño de $\mathcal{M}$, donde $\alpha = 0.75$.
		\begin{equation}
			R_{\text{seek}} = \alpha \cdot R_{\text{max}}
			\label{eq:Rseek}
		\end{equation}
		
		
		\item \textbf{Radio de medici\'{o}n}. Se describe un conjunto de radios de medici\'{o}n como $n_{r}$ con un incremento radial $dr$ (ecuaciones \ref{eq:nr} y \ref{eq:dr}).
		\begin{equation}
			n_{r} = 50 
			\label{eq:nr}
		\end{equation}
		
		\begin{equation}
			dr = \frac{r_{max}-r_{min}}{n_{r}-1}
			\label{eq:dr}
		\end{equation}
		
		Donde 
		
		\begin{equation}
			r_{min} = 0.15\, R_{\text{max}}, \qquad
			r_{max} = 20
			\label{eq:rmaxymin}
		\end{equation}
		
		
		\item \textbf{Radio de medici\'{o}n local.} Para cada centro de muestreo se especifica
		un centro local, v\'{e}ase la ecuaci\'{o}n \ref{eq:rloc}.
		\begin{equation}
			r_{loc} = \beta R_{\text{max}}, \qquad \beta = 0.15
			\label{eq:rloc}
		\end{equation} 
		
		$\beta$ limita el tamaño de las esferas a $15\%$ de $R_{max}$.
		
		\item \textbf{Centros de medici\'{o}n v\'{a}lidos.} 
		Un centro $\mathbf{r}_c$ puede usarse como centro s\'{o}lo si se cumple la condici\'{o}n \ref{eq:rc}. 
		
		\begin{equation}
			\|\mathbf{r}_c\| + r_{\mathrm{loc}} \le R_{\text{seek}}
			\label{eq:rc}
		\end{equation}
		La condici\'{o}n anterior asegura que la esfera de radio $r_{\text{loc}}$ quede completamente dentro de la regi\'{o}n de muestreo.
		\end{itemize}

	 
	 

	 \subsection{Paso 6 y 7 del diagrama de flujo general}
	 \label{subsec:paso6}
	 
	 En este paso se utiliza un criterio llamado densidad de part\'{i}culas promedio $\rho_{P}$ que tiene como objetivo determinar cuando un centro local $(r_c)$(ver paso 6 del diagrama de flujo presentado en la Fig. \ref{dfMolFractalDim}) no es adecuado para realizar la medici\'{o}n de la dimensi\'{o}n fractal masa. Por ejemplo, si una esfera de cierto radio centrada en alg\'{u}n punto de la prote\'{i}na tiene una $\rho_{P} < 0.034$ (en la secci\'{o}n \ref{subsec:dpp} se discute a detalle la raz\'{o}n del valor anterior), ser\'{i}a altamente probable que la esfera intersecte en una zona considerada como un hueco de la prote\'{i}na. Por lo tanto, un centro con tales caracter\'{i}sticas podr\'{i}a ser descartado hasta que se encuentre otro centro adecuado (ver bucle 6 $\longleftrightarrow$ 7 en la Figura \ref{dfMolFractalDim}). Visualmente este criterio se observa en la Figura \ref{fig:centrob}. 
	 	 \color{black}
	 
	 	\begin{figure}[H]
	 	\centering
	 	\includegraphics[width=0.5\linewidth]{graphs/centrob4.pdf}
	 	\caption{Diagrama 2D del c\'{a}lculo de la dimensi\'{o}n fractal de masa en prote\'{i}nas mediante \textit{molfractaldim}. (1) Se define un radio m\'{a}ximo en la prote\'{i}na ($R_{max}$) alrededor del centro de masa del sistema, (2) se determina si un centro local es adecuado ($\rho_{p} \geq 0.34$). Estas cantidades se utilizan para calcular la masa promedio de las part\'{i}culas $\bar{M}(r_k)$.}
	 	\label{fig:centrob}
	 \end{figure}
	 
	 
	 \clearpage
	 
	  Adem\'{a}s de lo anterior, $\rho_{P}$ permite refinar la medici\'{o}n de $\langle N(r_k) \rangle$, $\langle M(r_k) \rangle$ y $\langle Rg(r_k) \rangle$, que a su vez son las relaciones que se utilizan para medir las dimensiones fractales y as\'{i}, determinar si existe o no multifractalidad en estructuras proteicas. El criterio $\rho_{P}$ calcula dos valores clave: (1) El volumen de cada esfera de radio $r_k$ y (2) la densidad de part\'{i}culas promedio presente en cada esfera de radio $r_k$. El volumen de cada esfera se calcula usando el radio $r_k$ de las esferas definidas anteriormente y a partir de la ecuaci\'{o}n \ref{eq:volumen}:\\
	 
	 
	 \begin{equation}
	 	V(r_k) = \frac{4}{3} \pi r_{k}^{3}
	 	\label{eq:volumen}
	 \end{equation}
	 
	 La densidad de part\'{i}culas promedio se calcula a partir de la ecuaci\'{o}n \ref{eq:densidad}. Donde $\rho_{P}$ es la densidad de part\'{i}culas promedio presentes en la esfera de radio $r_k$ (ecuaci\'{o}n \ref{eq:densidad}). $\bar N(r_{k})$ es la cantidad promedio de part\'{i}culas presentes en una esfera de radio $r_k$. $V_{k}$ es el volumen de una esfera de radio $r_{k}$.
	 
	 \begin{equation}
	 	\rho_P = \frac{\bar N(r_{k})}{V_{k}}
	 	\label{eq:densidad}
	 \end{equation}
	 

	\subsection{Paso 8 del diagrama de flujo general}
	\label{subsec:paso8}

	Para cada radio $r_k$ en la malla $\{r_{\text{min}}, r_{\text{min}} + \Delta r, \ldots, r_{\text{max}}\}$, se definen las ecuaciones \ref{eq:N}, \ref{eq:M} y \ref{eq:Rg}.
	
	\begin{equation}
		N(\mathbf{r}_c, r_k) = \sum_{i=1}^{N_{\text{at}}} 
		\Theta\!\bigl(r_k - \|\mathbf{r}_i' - \mathbf{r}_c\|\bigr)
		\label{eq:N}
	\end{equation}
	
	\begin{equation}
		M(\mathbf{r}_c, r_k) = \sum_{i=1}^{N_{\text{at}}} 
		m_i \Theta\!\bigl(r_k - \|\mathbf{r}_i' - \mathbf{r}_c\|\bigr)
		\label{eq:M}
	\end{equation}
	
	\begin{equation}
		R_g(\mathbf{r}_c, r_k) = 
		\sqrt{\frac{\displaystyle\sum_{i=1}^{N_{\text{at}}} 
				\|\mathbf{r}_i' - \mathbf{r}_c\|^2  
				\Theta\!\bigl(r_k - \|\mathbf{r}_i' - \mathbf{r}_c\|\bigr)}
			{N(\mathbf{r}_c, r_k)}}
		\label{eq:Rg}
	\end{equation}
	
	Donde $\mathbf{r}_c$ es el centro de muestreo dentro del sistema $\mathcal{M}$,
	$\mathbf{r}_i'$ es la posici\'{o}n del part\'{i}cula $i$ respecto al centro de masa, 
	$m_i$ es la masa del part\'{i}cula $i$. $\Theta(x)$ es la funci\'{o}n de Heaviside que permite contar s\'{o}lo los \'{a}tomos dentro de la esfera, igual a $1$ si $x \ge 0$ y $0$ si $x < 0$, $N(\mathbf{r}_c, r_k)$ es n\'{u}mero de part\'{i}culas dentro de la esfera de radio $r_k$, $M(\mathbf{r}_c, r_k)$ es la masa total dentro de la esfera y 
	$R_g(\mathbf{r}_c, r_k)$ es radio de giro de las part\'{i}culas de la esfera.
	
	 	
 	\subsection{Paso 9 del diagrama de flujo general}
 		\label{subsec:paso9}
 	
 	\textbf{C\'{a}lculo de valores promedio sobre m\'{u}ltiples centros}. Se seleccionan aleatoriamente centros de muestreo en $\{\mathbf{r}_{c_j}\}$ (ve\'{a}se la ecuaci\'{o}n \ref{eq:Nmeas}) que puede modificarse a conteos menores (donde los promedios fluctuaran m\'{a}s) o conteos mayores (los promedios ser\'{a}n muy parecido entre s\'{i}). 
 	
 	
 	\begin{equation}
 		N_{\text{meas}} = 30  
 		\label{eq:Nmeas}
 	\end{equation}
 	
 	Por \'{u}ltimo, para calcular los valores de $\langle N(r_k) \rangle$, $\langle M(r_k) \rangle$ y $\langle Rg(r_k) \rangle$ se utilizan las ecuaciones \ref{eq:Np}, \ref{eq:Mp} y \ref{eq:Rgp}:
 	
 	\begin{equation}
 		\langle N(r_k) \rangle = \frac{1}{N_{\text{meas}}} 
 		\sum_{j=1}^{N_{\text{meas}}} N(\mathbf{r}_{c_j}, r_k)
 		\label{eq:Np}
 	\end{equation}
 	
 	\begin{equation}
 		\langle M(r_k) \rangle = \frac{1}{N_{\text{meas}}} 
 		\sum_{j=1}^{N_{\text{meas}}} M(\mathbf{r}_{c_j}, r_k)
 		\label{eq:Mp}
 	\end{equation}
 	
 	\begin{equation}
 		\langle R_g(r_k) \rangle = \frac{1}{N_{\text{meas}}} 
 		\sum_{j=1}^{N_{\text{meas}}} R_g(\mathbf{r}_{c_j}, r_k),
 		\label{eq:Rgp}
 	\end{equation}
 	
 	Donde $	N_{\text{meas}}$ es el n\'{u}mero de centros aleatorios sobre los que se promedia.
 	$\mathbf{r}_{c_j}$ es el vector de posici\'{o}n del j-\'{e}simo centro de medici\'{o}n y
 	$\langle\cdot\rangle$ es el promedio sobre todos los centros.	

 	
 	\subsection{Paso 10 del diagrama de flujo general}
 	\label{subsec:paso10}
 	
 	\textbf{Determinaci\'{o}n de la dimensi\'{o}n fractal.}	 
 	Para calcular la dimensi\'{o}n fractal de masa $D$, 
 	se asume un comportamiento de ley de potencia, 
 	ver ecuaci\'{o}n \ref{eq:leyp2}.
 	
 	\begin{equation}
 		\langle Z(r_{k}) \rangle \propto r^{D}_{k}
 		\label{eq:leyp2}
 	\end{equation}
 	
 	El exponente de escala $D$ relaciona $\langle Z(r) \rangle$ 
 	con $r_{k}$ donde $\langle Z(r) \rangle$ representa $\langle M(r_k) \rangle$,
 	 $\langle N(r_k) \rangle$ o $\langle Rg(r_k) \rangle$. 
 	Por \'{u}ltimo, se crea una gr\'{a}fica de $\log_{10}Z(r_k)$ vs $\log_{10}r_{k}$ 
 	(Figura \ref{fig:GrafD}) y se aplica una 
 	regresi\'{o}n lineal en escala logar\'{i}tmica.
 	
	\begin{figure}[H]
		\hspace{-0.3cm} 
		\begin{minipage}{0.49\textwidth}
			\centering
			\includegraphics[width=\linewidth]{graphs/molmass/Zvsr.pdf}
		\end{minipage}
		\hspace{0.2cm}
		\begin{minipage}{0.49\textwidth}
			\centering
			\includegraphics[width=\linewidth]{graphs/molmass/Zvsr-Rg.pdf}
		\end{minipage}
		
		\caption{
			Gr\'{a}ficos obtenido mediante el programa \textit{molfractaldim} de $\log_{10}r$ vs $\log_{10}Z(r)$ donde $\log_{10}Z(r)$ 
			corresponde a $\langle M(r_k) \rangle$, $\langle N(r_k) \rangle$ 
			o $\langle Rg(r_k) \rangle$. }
		\label{fig:GrafD}
	\end{figure}

 	
 	\section{Metodolog\'{i}a}
 	\label{sec:paso8}
 	
 	\subsection{Densidad de part\'{i}culas promedio $\rho_{P}$}
 	\label{subsec:dpp}
 	
	La implementaci\'{o}n del criterio, densidad de part\'{i}culas 
 	promedio $\rho_{P}$ (ver subsecci\'{o}n \ref{subsec:paso6}) 
 	fue realizada seleccionando 79 prote\'{i}nas, v\'{e}ase las Tablas \ref{Tabla:ids79(1)} y \ref{Tabla:ids79(2)} (provenientes del \textit{Protein Data Bank} con fecha de acceso: agosto de 2024) y con el uso del programa \emph{molmassfractaldim} 
 	(sin la aplicaci\'{o}n del bucle 6 $\longleftrightarrow$ 7 
 	del diagrama de flujo de la Figura \ref{dfMolFractalDim}) 
 	se proces\'{o} cada prote\'{i}na mencionada. 
	
	Una vez completada esta fase, se extrajeron los valores 
 	de los radios $r^*{=}$ 5, 10, 15, 20, 25, 30 $\textup{\r{A}}$
 	 junto con el n\'{u}mero de part\'{i}culas promedio $\langle N(r^{*}) \rangle$ 
 	  de las datos generados a partir del programa \emph{molmassfractaldim}.
 	   Con esta informaci\'{o}n, se procedi\'{o} al c\'{a}lculo de vol\'{u}menes y densidades 
 	   de part\'{i}culas promedio presentes en las 79 prote\'{i}nas. 
	
 	\begin{table}[H]
 		\centering
 		\begin{footnotesize}
 			\begin{tabular}{||l|ll||l|ll||}
 				\multicolumn{6}{c}{}                                   \\
 				\hline
 				ldx & IdPDB & Nombre de prote\'{i}na       & ldx & IdPDB & Nombre de prote\'{i}na        \\ 
 				\hline
 				1   & 11gs  & Glutati\'{o}n S-transferasa      & 21 & 1a27  & 17$\beta$-hidroxiesteroide    \\
 				2   & 12ca  & Anhidrasa carb\'{o}nica II       & 22 & 1a28  & Receptor de progesterona      \\
 				3   & 13gs  & Glutati\'{o}n S-transferasa      & 23 & 1a2b  & Prote\'{i}na transformante RhoA   \\
 				4   & 14gs  & Glutati\'{o}n S-transferasa      & 24 & 1a2c  & Trombina                      \\
 				5   & 16gs  & Glutati\'{o}n S-transferasa P    & 25 & 1a31  & Topoisomerasa I               \\
 				6   & 1a00  & Hemoglobina                  & 26 & 1a35  & Topoisomerasa I               \\
 				7   & 1a01  & Hemoglobina                  & 27 & 1a36  & ADN topoisomerasa I           \\
 				8   & 1a02  & Factor nuclear de c\'{e}lulas    & 28 & 1a3b  & Trombina                      \\
 				9   & 1a07  & Tirosina quinasa src         & 29 & 1a3e  & Trombina                      \\
 				10  & 1a08  & Tirosina quinasa src         & 30 & 1a3n  & Hemoglobina                   \\
 				11  & 1a09  & Tirosina quinasa src         & 31 & 1a3o  & Hemoglobina                   \\
 				12  & 1a0l  & Triptasa beta-1              & 32 & 1a3q  & Factor nuclear $\kappa$-B     \\
 				13  & 1a0u  & Hemoglobina                  & 33 & 1a42  & Anhidrasa carb\'{o}nica II        \\
 				14  & 1a0z  & Hemoglobina                  & 34 & 1a46  & Trombina                      \\
 				15  & 1a12  & Regulador de cromosoma       & 35 & 1a4i  & Dihidrofolato reductasa       \\
 				16  & 1a15  & Factor 1-$\alpha$ de estroma & 36 & 1a4r  & Prote\'{i}na de uni\'{o}n G25K        \\
 				17  & 1a17  & Fosfatasa ser/thr            & 37 & 1a4y  & Inhibidor de ribonucleasa     \\
 				18  & 1a1m  & Ant\'{i}geno  HLA clase I        & 38 & 1a52  & Receptor de estr\'{o}geno         \\
 				19  & 1a1o  & Ant\'{i}geno  HLA clase I        & 39 & 1a66  & N\'{u}cleo NFATC1                 \\
 				20  & 1a22  & Hormona de crecimiento       & 40 & 1a7a  & Adenosilhomocisteinasa        \\ \hline
 			\end{tabular}
 		\end{footnotesize}
 		\caption{\'{I}ndice de prote\'{i}nas (ldx), identificadores del \emph{Protein Data Bank} (IdPDB) y nombre de las prote\'{i}nas (Parte 1).}
 		\label{Tabla:ids79(1)}
 	\end{table}
 	
 	\begin{table}[H]
 		\centering
 		\begin{footnotesize}
 			\begin{tabular}{||l|ll||l|ll||}
 				\multicolumn{6}{c}{}                                         \\
 				\hline
 				ldx & IdPDB & Nombre de prote\'{i}na               & ldx & IdPDB & Nombre de prote\'{i}na           \\ 
 				\hline
 				41  & 1a7c  & Inhibidor del activador plasmin\'{o}geno & 61 & 1awb  & Mioinositol monofosfatasa        \\
 				42  & 1a8m  & Factor de necrosis tumoral           & 62 & 1azx  & Antitrombina III                 \\
 				43  & 1a9b  & Ant\'{i}geno HLA-B                       & 63 & 1b09  & Regulador de transcripci\'{o}n CRP   \\
 				44  & 1a9e  & Ant\'{i}geno HLA-B                       & 64 & 1b3e  & Transferrina s\'{e}rica              \\
 				45  & 1a9w  & Hemoglobina                          & 65 & 1b3o  & Inosina monofosfato              \\
 				46  & 1aax  & Tirosina fosfotasa 1B                & 66 & 1b41  & Acetilcolinesterasa              \\
 				47  & 1ab2  & Tirosina quinasa C-ABL               & 67 & 1b47  & CBL                              \\
 				48  & 1abw  & Hemoglobina                          & 68 & 1b59  & Metionina aminopeptidasa         \\
 				49  & 1aby  & Hemoglobina                          & 69 & 1b5g  & Trombina $\alpha$                \\
 				50  & 1agn  & Alcohol deshidrogenasa               & 70 & 1b86  & Hemoglobina                      \\
 				51  & 1ah1  & CTLA-4                               & 71 & 1ba8  & Trombina                         \\
 				52  & 1ald  & Aldolasa fructosa-bifosfato          & 72 & 1bab  & Hemoglobina                      \\
 				53  & 1ant  & Antitrombina-III                     & 73 & 1bbb  & Hemoglobina                      \\
 				54  & 1aos  & Argininosuccinato liasa              & 74 & 1bcn  & Interleucina-4                   \\
 				55  & 1ap5  & Super\'{o}xido dismutasa                 & 75 & 1bhg  & Beta-glucuronidasa               \\
 				56  & 1apy  & Aspartilglucosaminidasa              & 76 & 1bhs  & 17-$\beta$-deshidrogenasa        \\
 				57  & 1apz  & Aspartilglucosaminidasa              & 77 & 1bhx  & Trombina                         \\
 				58  & 1aud  & U1A 102                              & 78 & 6oj0  & Prote\'{i}na estructural VP4         \\
 				59  & 1auk  & Arilsulfatasa A                      & 79 & 7khw  & Translocon EspA                  \\
 				60  & 1avo  & Regulador 11S                        &    &       &  \\ \hline
 			\end{tabular}
 		\end{footnotesize}
 		\caption{\'{I}ndice de prote\'{i}nas (ldx), identificadores del \emph{Protein Data Bank} (IdPDB) y nombre de las prote\'{i}nas (Parte 2).}
 		\label{Tabla:ids79(2)}
 	\end{table}

 
 	\clearpage
 	
 	
 	A continuaci\'{o}n, se cre\'{o} una gr\'{a}fica (Figura \ref{index-vs-density}) para observar la distribuci\'{o}n 
 	de los valores de $\rho_{P}$ en funci\'{o}n del \'{i}ndice de cada
 	prote\'{i}na ldx de las Tablas \ref{Tabla:ids79(1)} y \ref{Tabla:ids79(2)},
 	 as\'{i} como la densidad doblemente promediada, es decir, $\nu\equiv\bar{\rho_P}$ 
 	 y la desviaci\'{o}n est\'{a}ndar. Tanto $\nu$ como $\sigma$ se calcularon
 	  sobre los distintos $\rho_P(r^*)$.
 	
 	\begin{figure}[H]
 		\centering
 		\includegraphics[width=\linewidth]{graphs/ldx-dp.pdf}
 		\caption{Distribuci\'{o}n de la densidad de 
 			part\'{i}culas promedio ($\rho_P$) en funci\'{o}n del 
 			\'{i}ndice de prote\'{i}nas (ldx) para
 			las prote\'{i}nas de las Tablas 
 			\ref{Tabla:ids79(1)} y \ref{Tabla:ids79(2)}. La l\'{i}nea negra horizontal 
 			discontinua representa la media ($\nu$) de   
 			$\rho_P$, mientras que las l\'{i}neas rojas horizontales discontinuas indican los l\'{i}mites de dos desviaciones est\'{a}ndar (\(\nu \pm 2\sigma\)) de ($\rho_P$).}
 		\label{index-vs-density}
 	\end{figure}
 	
 	
 	De la Figura \ref{index-vs-density} se observ\'{o} que la mayor\'{i}a de las prote\'{i}nas de las Tablas \ref{Tabla:ids79(1)} y \ref{Tabla:ids79(2)}  presentan una $\rho_P$ entre 0.002 y 0.070. 
 	Se distingue, adem\'{a}s, una distribuci\'{o}n moderadamente sesgada de $\rho_P$ hacia valores de radios menores ($r^* \leq 20$). Es decir, los valores de \(\rho_P\) se encuentran cerca de la media (0.034) especialmente con valores de $r^*{=}$ 5, 10, 15, 20. Pero existe un rango considerable que se extiende hacia valores menores a partir de $r^*$= 25, 30.
 	
 	
 	Se puede inferir que $\rho_P$ disminuye conforme aumenta el radio, cuyo comportamiento resulta consistente con el esperado en sistemas f\'{i}sicos, donde las part\'{i}culas tienden a estar m\'{a}s densamente distribuidas en las regiones pr\'{o}ximas al centro de un volumen esf\'{e}rico.
 	
	\clearpage
	
 	Es importante mencionar que pocos valores caen fuera del intervalo $\nu \pm 2\sigma$ (11 valores con $r^*$ = 5, 10, 15) por lo tanto, las variaciones significativas son limitadas y pueden representar anomal\'{i}as o efectos f\'{i}sicos menos frecuentes en tales prote\'{i}nas.
 	
 	De la informaci\'{o}n anterior, rescatamos que al analizar una prote\'{i}na arbitraria, se puede utilizar el valor de la densidad de part\'{i}culas $\rho_{p} \geq 0.034$ como criterio para determinar cu\'{a}ndo un centro no es adecuado para realizar la medici\'{o}n de la dimensi\'{o}n fractal de masa. 
 	
 	
	\subsection{Pruebas de fractalidad en prote\'{i}nas}
	\label{subsec:pfp}

 	
 	El an\'{a}lisis fractal fue realizado tomando como punto de partida 9 de las 79 estructuras que se utilizaron anteriormente aplicando la restricci\'{o}n de la densidad de part\'{i}culas  promedio ($\rho_{p} \geq 0.034$) como criterio para determinar cuando un centro no es adecuado para realizar el conteo de part\'{i}culas y posteriormente, la cuantificaci\'{o}n de la masa como funci\'{o}n del radio de las esferas. Adem\'{a}s, se hizo uso del programa de din\'{a}mica molecular \textit{GROMACS}\cite{Lemkul2024, Abraham2015} que permite realizar simulaciones moleculares, con el fin de comparar diferentes estados de los sistemas proteicos. Los pasos esenciales de este proceso fueron los siguientes: (1) Preparaci\'{o}n de la topolog\'{i´}a del sistema. (2) Creaci\'{o}n de la caja y solvataci\'{o}n, (3) adici\'{o}n de iones para la neutralizaci\'{o}n de la carga, (4) minimizaci\'{o}n de la energ\'{i}a del sistema, (5) equilibrio del sistema, (6) simulaci\'{o}n de producci\'{o}n y (7) resultados.
 	
 	
 	
 	Hecho lo anterior, se extrajeron las estructuras correspondientes a la \'{u}ltima 
 	geometr\'{i}a de los siguientes procesos, para una comparaci\'{o}n posterior.
 	
 	\begin{enumerate}
 		\item Despu\'{e}s de adicionar \'{a}tomos de hidr\'{o}geno al sistema proteico. 
 		\item Tras finalizar el proceso de minimizaci\'{o}n de la energ\'{i}a.
 		\item Luego de completarse la etapa de equilibrio del sistema.
 		\item Completada la din\'{a}mica molecular durante un 1 ns.
 	\end{enumerate}
 	
 	
 	\subsection{Elecci\'{o}n de sistemas}
 	\label{subsec:eleccionsis}
	
		La selecci\'{o}n de los sistemas se realiz\'{o} aleatoriamente (utilizando como punto de partida las 79 pote\'{i}nas de las Tablas \ref{Tabla:ids79(1)}, \ref{Tabla:ids79(2)} y creando un nuevo \'{i}ndice de prote\'{i}nas para cada una de ellas), bajo la condici\'{o}n de que todos sistemas contaran con la secuencia completa de amino\'{a}cidos y evitando componentes externos como ligandos o disolventes que introducir\'{i}an ruido en el c\'{a}lculo de la dimensi\'{o}n fractal masa intr\'{i}nseca de las prote\'{i}nas. Los sistemas analizados se listan en la Tabla \ref{Tabla:ids9}.
 	
 	\begin{table}[H]
 		\centering
 		\begin{tabular}{||lllc||}
 			\multicolumn{4}{l}{} \\ 
 			\hline
 			ldx & IdPDB & Nombre de prote\'{i}na & N\'{u}mero de \'{a}tomos \\
 			\hline
 			 1  & 1a2b & Transformadora RHOA & 2831 \\
 			 2 & 1b3e & Transferrina s\'{e}rica & 5037 \\
 			 3 & 11gs & Glutati\'{o}n s-transferasa & 6536 \\ 
 			 4 & 1auk & Arilsulfatasa A & 7086 \\
 			 5 & 1a8m & Factor de necr\'{o}sis tumoral $\alpha$ & 7104 \\
 			 6 & 1a52 & Receptor de estr\'{o}geno & 7765 \\
 			 7 & 1a3n & Hemoglobina humana desoxidante cadena $\alpha$ & 8734 \\
 			 8 & 1a9w & Hemoglobina embrionaria cadena $\alpha$ & 8820 \\
 			 9 & 7khw & Translocon EspA & 131200 \\
 			\hline
 		\end{tabular}
 		\caption{Nuevo \'{i}ndice de prote\'{i}nas (ldx), Identificadores del \emph{Protein Data Bank} (PDB), nombre de la prote\'{i}na y n\'{u}mero de \'{a}tomos presentes en cada estructura.}
 		\label{Tabla:ids9}
 	\end{table}
 	
 	
 
 	\clearpage
	
	\subsection{Elecci\'{o}n de medida de an\'{a}lisis}
	\label{eleccionMN}
	
	 Nuestro estudio se ha enfocado en analizar la dimensi\'{o}n fractal de masa $M(r)$ debido a las siguientes razones:  
	 
	 \begin{enumerate}
	 	\item La dimensi\'{o}n fractal de masa $M(r)$ captura la distribuciz\'{o}n real de masa en un sistema. 
	 	
	 	En una prote\'{i}na se presentan \'{a}tomos que tienen masas at\'{o}micas muy diferentes, adem\'{a}s de localizadas y centradas en los n\'{u}cleos (ejemplo de lo anterior son: H $\sim$1 g mol$^{-1}$, C $\sim$12 g mol$^{-1}$, N $\sim$14 g mol$^{-1}$, O $\sim$16 g mol$^{-1}$, S $\sim$32 g mol$^{-1}$). Ignorar esta heterogeneidad, como hace la dimensi\'{o}n fractal del n\'{u}mero de part\'{i}culas $N(r)$, arrojar\'{i}a un an\'{a}lisis incompleto o diferente. Dos posibles escenario al realizar una medida de $M(r)$ y $N(r)$ son los siguientes:
	 	 
	 	 \begin{itemize}
	 	 	\item Una zona con gran cantidad de \'{a}tomos de hidr\'{o}geno (baja masa) contribuir\'{a} mayormente a la medida de $N(r)$ pero menor a la medida $M(r)$.
	 	 	
	 	 	\item Una regi\'{o}n densa con \'{a}tomos de azufre o con la presencia de un grupo hemo contribuir\'{a} masivamente a $M(r)$ pero no a $N(r)$.
	 	 \end{itemize}
	 	 
		\item Relaci\'{o}n con multifractalidad.
		
		La multifractalidad surge de la heterogeneidad en un objeto de an\'{a}lisis. Al usar la masa como medida $M(r)$, se asegura una heterogeneidad adicional y f\'{i}sicamente significativa respecto a solo contar part\'{i}culas ($N(r)$, es decir, tratar a todos los \'{a}tomos por igual). Por lo tanto, la medida de $M(r)$ incluye informaci\'{o}n de las masas at\'{o}micas.
	 	 
	 \end{enumerate}
	
	\clearpage
	
	\subsection{Determinaci\'{o}n de $M(r)$}
	\label{subsec:detM(r)}	
	
	Se calcul\'{o} la masa promedio de las part\'{i}culas 
	contenidas dentro de las esferas de radio $r_k$, definida 
	como $\langle M(r_{k}) \rangle$. 
	Para llevar a cabo este an\'{a}lisis, se hizo uso de la 
	herramienta \textit{molfractaldim}. Ajustando dos par\'{a}metros
	clave: (1) $N_{meas}$ (ecuaci\'{o}n \ref{eq:Nmeas}) que define cu\'{a}ntos 
	valores se promedian para obtener el valor de $\langle M(r_{k}) \rangle$,
	cuando $N_{meas} = 1$, los valores promediados presentan una mayor 
	variabilidad, como puede observarse en la primera gr\'{a}fica de la 
	Figura \ref{fig:Nmeasp}. En cambio, al aumentar el n\'{u}mero de mediciones 
	a $N_{meas} = 200$, las fluctuaciones disminuyen considerablemente,
	 y los valores promediados tienden a ser muy similares entre s\'{i}, como se 
	 muestra en la segunda gr\'{a}fica de la Figura \ref{fig:Nmeasp}.
	 Y (2) $r_{max}$ (ecuaci\'{o}n \ref{eq:rmaxymin}) que determina 
	cu\'{a}ntos radios $r_{k}$ se toman para la medici\'{o}n.
	
	En este an\'{a}lisis  la elecci\'{o}n de $N_{meas} = 200 $ y $r_{max} = 200$ se hizo para asegurar resultados 
	representativos y detallados. Este procedimiento se aplic\'{o} en cada uno 
	de los sistemas de la Tabla~\ref{Tabla:ids9}, correspondientes a 
	los pasos descritos en la secci\'{o}n \ref{subsec:pfp}.
	

	\begin{figure}[H]
		\hspace{-0.3cm} 
		\begin{minipage}{0.49\textwidth}
			\centering
			\includegraphics[width=\linewidth,page=1]{graphs/PDBs/Tubb4/TubMutA=1.pdf}
		\end{minipage}
		\hspace{0.2cm}
		\begin{minipage}{0.49\textwidth}
			\centering
			\includegraphics[width=\linewidth,page=1]{graphs/PDBs/Tubb4/TubMutA=200.pdf}
		\end{minipage}
		
		\caption{
			Gr\'{a}ficos de $\log_{10}r$ vs $\log_{10}N(r)$ correspondientes la prote\'{i}na \textit{Tubulina mutada} despu\'{e}s de una din\'{a}mica molecular de 10 ns, estableciendo el n\'{u}mero de medidas en  $N_{means}$ = 1 y 200 (izquierda a derecha). Donde $N_{means}$ establece el n\'{u}mero de medidas para obtener $\langle N(r_k) \rangle$, $\langle M(r_k) \rangle$ y $\langle Rg(r_k) \rangle$.}
		\label{fig:Nmeasp}
	\end{figure}
	
	
	\begin{comment}
		Determinaci\'{o}n de la masa promedio de las part\'{i}culas $\bar M(r_{i})$ dentro de cada radio $r_{i}$.Donde:
		
		\begin{itemize}
			\item $r_{i}$ es el radio de una esfera.
			\item $\bar M(r_{i})$ es la masa promedio de las part\'{i}culas promedio contenidas dentro una esfera de radio $r$.
		\end{itemize}
		
		A partir del programa \textit{molmassfractaldim} usando las restriciones $-a$ y $-n$ que modifican el n\'{u}mero promedios para  $M(r)$ y establecen el n\'{u}mero de radios para $M(r)$ vs $r$.  Este procedimiento se aplic\'{o} a cada uno de los 9 archivos listados en la Tabla \ref{Tabla:ids}, correspondientes a cada paso descrito en la metodolog\'{i´}a. 
	\end{comment}
	
	
	\clearpage
	
	\subsection{An\'{a}lisis multifractal}
	\label{subsec:anaD}
	
	Se realiz\'{o} un an\'{a}lisis multifractal sobre los datos extra\'{i}dos del paso anterior, para calcular la dimensi\'{o}n fractal de masa $D$, donde el exponente de escala $b$  relaciona la masa  promedio $\langle M(r_{k}) \rangle$ con el radio $r_{k}$ mediante la ecuaci\'{o}n \ref{eq:leyp3}.
	\begin{equation}
		\langle M(r_{k}) \rangle \sim r_{k}^b
		\label{eq:leyp3}
	\end{equation}
	
	
	Para un sistema multifractal, $b$ var\'{i}a seg\'{u}n la escala espacial, reflejando heterogeneidad en la distribuci\'{o}n de masa.
	\begin{itemize}
	%	\item \textbf{Coeficiente de determinaci\'{o}n $R^{2}$}: \\ Es una medida de correlaci\'{o}n lineal entre $log_{10}M(r)$ y $log_{10}r$.
		\item \textbf{Pendiente $b$}:
		\begin{itemize}
			\item Si $b$ muestra un \'{u}nico valor en el intervalo general 1--20~ $\textup{\r{A}}$: Existe un comportamiento monofractal (estructura homog\'{e}nea).
			\item Si $b$  tiene distintos valores en el intervalo general de 1--20~ $\textup{\r{A}}$: Existe multifractalidad (estructura heterog\'{e}nea).
		\end{itemize}
	\end{itemize}
	
	A continuaci\'{o}n, se llevaron a cabo m\'{u}ltiples regresiones lineales sobre cada archivo de datos num\'{e}ricos, en intervalos donde era altamente probable encontrar distintos valores de dimensi\'{o}n fractal. Estos intervalos fueron: 1 a 2~$\textup{\r{A}}$, 2 a 6~$\textup{\r{A}}$ y 6 a 20~$\textup{\r{A}}$ \cite{Enright2005, Liu2020}.	
	
	
	
	La selecci\'on del intervalo 2--6~$\textup{\r{A}}$ utilizado para la medici\'on o implementaci\'on de ajustes lineales en el an\'alisis multifractal tiene su origen en el di\'ametro promedio de los amino\'acidos ($d_p$) reportado por Siyuan Liu \textit{et al.} \cite{Liu2020} siendo $d_p \approx$ 4,188 $\textup{\r{A}}$. Posteriomente, la elecci\'{o}n de los intervalos parciales 1--2~$\textup{\r{A}}$ y 6--20~$\textup{\r{A}}$ fueron elegidos heur\'{i}sticamente, observando las gr\'{a}ficas de $\log_{10}r$ vs $\log_{10}M(r)$ de cada prote\'{i}na de la Tabla \ref{Tabla:ids9}.
	
	\section{Discusi\'{o}n de resultados}	
	\label{sec:disresult}
	
	\begin{figure}[H]
		\centering
		\begin{subfigure}{0.49\textwidth}
			\centering
			\includegraphics[width=\linewidth,page=1]{graphs/PDBs/Dvsldx/DvsaddH.pdf}
			\caption{(1)}
		\end{subfigure}
		\hfill
		\begin{subfigure}{0.49\textwidth}
			\centering
			\includegraphics[width=\linewidth,page=1]{graphs/PDBs/Dvsldx/DvsEm.pdf}
			\caption{(2)}
		\end{subfigure}
		
		\vspace{0.3cm}
		
		\begin{subfigure}{0.49\textwidth}
			\centering
			\includegraphics[width=\linewidth,page=1]{graphs/PDBs/Dvsldx/DvsEq.pdf}
			\caption{(3)}
		\end{subfigure}
		\hfill
		\begin{subfigure}{0.49\textwidth}
			\centering
			\includegraphics[width=\linewidth,page=1]{graphs/PDBs/Dvsldx/DvsDm.pdf}
			\caption{(4)}
		\end{subfigure}
		
		\caption{Valores de dimensi\'{o}n fractal de masa $D$ vs identificadores de prote\'{i}nas (IdPDB) para las prote\'{i}nas ilustradas en la Tabla \ref{Tabla:ids9}, correspondientes a cuatro etapas de procesamiento: (1) Adici\'{o}n de \'{a}tomos de hidr\'{o}geno al sistema proteico, (2) al minimizar la energ\'{i´}a de la estructura molecular, (3) equilibrando el sistema bajo condiciones termodin\'{a}micas controladas y (4) despu\'{e}s de una din\'{a}mica molecular de 1 ns. Las l\'{i}neas azules, verdes y rojas representan $D$ en los intervalos de 1 a 2~$\textup{\r{A}}$, 2 a 6~$\textup{\r{A}}$ y 6 a 20~$\textup{\r{A}}$, respectivamente.}
		\label{fig:Df-general}
	\end{figure}
	
	\clearpage
	
	\begin{figure}[H]
		\subsection{Efecto de iones en la dimensi\'{o}n fractal de masa}
		\label{subsec:EfectoIones}	
		\centering
		\begin{subfigure}{0.49\textwidth}
			\centering
			\includegraphics[width=\linewidth,page=1]{graphs/PDBs/7khw/ions/7khwEq-wions.pdf}
			\caption{(1)}
		\end{subfigure}
		\hfill
		\begin{subfigure}{0.49\textwidth}
			\centering
			\includegraphics[width=\linewidth,page=1]{graphs/PDBs/7khw/ions/7khw1ns-wions.pdf}
			\caption{(2)}
		\end{subfigure}
		\caption{Gr\'{a}ficos de $\log_{10}r$ vs $\log_{10}M(r)$ correspondientes a dos etapas de procesamiento de la prote\'{i}na \textit{Translocon EsA}  de la Tabla \ref{Tabla:ids9} (con iones Na$^{+}$ en su estructura) despu\'{e}s de (1) equilibrar el sistema bajo condiciones termodin\'{a}micas controladas y (2) de una din\'{a}mica molecular de 1 ns. Los valores de $b$ en cada gr\'{a}fica representan la dimensi\'{o}n fractal de masa en los intervalos 1 a 2~$\textup{\r{A}}$, 2 a 6~$\textup{\r{A}}$ y 6 a 20~$\textup{\r{A}}$.}
		\label{fig:7khw-wions}
	\end{figure}
	
	
	\begin{figure}[H]
		\centering
		\begin{subfigure}{0.49\textwidth}
			\centering
			\includegraphics[width=\linewidth,page=1]{graphs/PDBs/7khw/ions/7khwEq-oions.pdf}
			\caption{(1)}
		\end{subfigure}
		\hfill
		\begin{subfigure}{0.49\textwidth}
			\centering
			\includegraphics[width=\linewidth,page=1]{graphs/PDBs/7khw/ions/7khw1ns-oions.pdf}
			\caption{(2)}
		\end{subfigure}
		\caption{Gr\'{a}ficos de $\log_{10}r$ vs $\log_{10}M(r)$ correspondientes a dos etapas de procesamiento de la prote\'{i}na \textit{Translocon EsA} de la Tabla \ref{Tabla:ids9} (sin iones Na$^{+}$ en su estructura) de la Tabla \ref{Tabla:ids9} despu\'{e}s de (1) equilibrar el sistema bajo condiciones termodin\'{a}micas controladas y (2) de una din\'{a}mica molecular de 1 ns. Los valores de $b$ en cada gr\'{a}fica representan la dimensi\'{o}n fractal de masa en los intervalos 1 a 2~$\textup{\r{A}}$, 2 a 6~$\textup{\r{A}}$ y 6 a 20~$\textup{\r{A}}$.}
		\label{fig:7khw-oions}
	\end{figure}
	
	\begin{table}[H]
		\centering
		\begin{tabular}{lllSS}
			\multicolumn{5}{c}{\textbf{Proteina Translocon EsA (idPDB: 7khw)}} \\
			\midrule
			\textbf{Estado de la prote\'{i}na} & \textbf{Etapa} & \textbf{Intervalo (\AA)} & \textbf{$R^{2}$} & \textbf{Pendiente ($b$)} \\
			\midrule
			\multirow{6}{*}{\centering Con Na$^{+}$}
			&            & 1-2 & 0.933264 & 1.67544 \\
			& Equilibrio & 2-6 & 0.997049 & 2.58655 \\
			&            & 6-20 & 0.999965 & 2.76119 \\
			\cmidrule(lr){2-5}
			&                    & 1-2 & 0.938494 & 1.59519 \\
			& Din\'{a}mica (1ns) & 2-6 & 0.997471 & 2.5893 \\
			&                    & 6-20 & 0.999977 & 2.77564 \\
			
			\cmidrule(lr){2-5}
			\multirow{6}{*}{\centering Sin Na$^{+}$}
			&            & 1-2 & 0.930791 & 1.52684 \\
			& Equilibrio & 2-6 & 0.996961 & 2.60358 \\
			&            & 6-20 & 0.999988 & 2.79052 \\
			\cmidrule(lr){2-5}
			&                    & 1-2 & 0.929409 & 1.59215 \\
			& Din\'{a}mica (1ns) & 2-6 & 0.99807 & 2.64238 \\
			&                    & 6-20 & 0.999938 & 2.78170 \\
			\bottomrule
		\end{tabular}
		\caption{Resumen de datos de las regresiones lineales de $\log_{10}r$ vs $\log_{10}M(r)$ de las Figuras \ref{fig:7khw-wions} y \ref{fig:7khw-oions}.}
		\label{tab:tab-7khw}
	\end{table}
	
	En las Figuras \ref{fig:7khw-wions} y \ref{fig:7khw-oions}, se compara la distribuci\'{o}n de masa de la prote\'{i}na \textit{Translocon EsA} (7khw) en dos etapas: Tras el equilibrio estructural y luego de una din\'{a}mica molecular de 1~ns. La Figura \ref{fig:7khw-wions} considera la estructura proteica con iones presentes, revelando tres pendientes en la regresi\'{o}n de $\log_{10}r$ vs $\log_{10}M(r)$, lo que sugiere un comportamiento multifractal en 3 intervalos. Sin embargo, en la Figura \ref{fig:7khw-oions}, 250 iones de sodio que no pertenec\'{i}an a la estructura intr\'{i}nseca de la prote\'{i}na fueron eliminados, con el fin de validar la reproducibilidad de los resultados. Sorprendentemente, en el \'{u}ltimo caso solo se observaron dos pendientes, aunque la estructura original conten\'{i}a  un excedente de 250 \'{a}tomos de sodio (aproximadamente el 0.19\% comparado con los 131,200 \'{a}tomos presentes en la estructura proteica). Por lo tanto, esta comparaci\'{o}n muestra que la estimaci\'{o}n de la dimensi\'{o}n fractal de masa es sensible a modificaciones estructurales m\'{i}nimas. %Estos resultados respaldan el uso potencial de dicha medida como herramienta de caracterizaci\'{o}n estructural en prote\'{i}nas. 
	
	
	\subsection{Caso particular de estudio:tubulinas}
	\label{subsec:tubulinas}
	
	\begin{figure}[H]
		\hspace{-0.3cm} 
		\begin{minipage}{0.49\textwidth}
			\centering
			\includegraphics[width=\linewidth,page=1]{graphs/PDBs/Tubb4/TubNat10ns.pdf}
		\end{minipage}
		\hspace{0.2cm}
		\begin{minipage}{0.49\textwidth}
			\centering
			\includegraphics[width=\linewidth,page=1]{graphs/PDBs/Tubb4/TubMut10ns.pdf}
		\end{minipage}
		
		
		\caption{
			Gr\'{a}ficos de $\log_{10}r$ vs $\log_{10}M(r)$ correspondientes a dos prote\'{i}nas diferentes: \textit{Tubulina nativa} (alanina) y \textit{Tubulina mutada} (treonina) despu\'{e}s de una din\'{a}mica molecular de 10 ns. Los valores de $b$ en cada gr\'{a}fica representan la dimensi\'{o}n fractal de masa en los intervalos 1 a 2~$\textup{\r{A}}$, 2 a 6~$\textup{\r{A}}$ y 6 a 20~$\textup{\r{A}}$.}
		\label{fig:Tubs}
	\end{figure}
	
	
	La comparaci\'{o}n presentada en la Figura~\ref{fig:Tubs} examina dos variantes estructurales: (1) Una prote\'{i}na nativa con 6,812 \'{a}tomos (estructura de referencia) y (2) una variante mutada con 6,816 \'{a}tomos (reemplazo de un residuo aminoac\'{i}dico).
	
	Los resultados obtenidos demostraron diferencias en las pendientes de la regresi\'{o}n lineal analizadas espec\'{i}ficamente en la etapa final del procesamiento (despu\'{e}s de una din\'{a}mica molecular de 10 ns). Esta diferencia estructural se observa a pesar de la alta similitud global entre ambas variantes.
	La detecci\'{o}n de estas diferencias sutiles, valida la sensibilidad del m\'{e}todo propuesto para discriminar modificaciones estructurales m\'{i}nimas en sistemas proteicos.
	
	Con base en la evidencia discutida en esta subsecci\'{o}n y las anteriores (subsecci\'{o}n \ref{subsec:DFE} a \ref{subsec:tubulinas}), se puede establecer que las prote\'{i}nas, en general, presentan un comportamiento multifractal. Este fen\'{o}meno se manifiesta en intervalos espec\'{i}ficos, los cuales han sido caracterizados en el an\'{a}lisis. Las prote\'{i}nas restantes del estudio exhiben un comportamiento an\'{a}logo. Para una revisi\'{o}n detallada, se remite al lector al Ap\'{e}ndice \ref{chap:9P} para examinar los detalles particulares de cada sistema.
	
	
	\color{black}