\chapter{Resultados y discusión}

La esencia de esta tesis es desarrollar un programa de cómputo científico para explorar la posible aplicación del análisis multifractal en estructuras proteicas. Este programa podría ser una herramienta útil para caracterizar las distintas etapas del plegamiento proteico, describiéndolas en términos de la dinámica de formación de agregados estructurales, lo que permitiría una comprensión más profunda de estos procesos. En consecuencia, es necesaria una explicación detallada de todo esto. Por lo que he dividido esta descripción en dos secciones; el algoritmo y su uso para la obtención de resultados. 
 
 \clearpage
 
\section{Algoritmo (diseño del programa)}

 El procedimiento general del programa \textbf{\textit{molmassfractaldim}} se describe en el diagrama de flujo de la figura \ref{dfMolFractalDim}), cuyos pasos se explican a detalle en las siguientes subsecciones.
 
 
 \begin{figure}[h!]
 	\begin{center}
 		\includegraphics[width=\textwidth]{graphs/dfMolFractalDim}
 		\caption{Diagrama de flujo general del programa \textit{molmassfractaldim}.}
 		\label{dfMolFractalDim}
 	\end{center}
 \end{figure}
 

 
 \subsection {Pasos 1, 2, 3 y 4 del diagrama de flujo general}
 
	\begin{itemize}
		\item \textbf{Paso 1}:  El programa \textbf{\textit{molmassfractaldim}} inicia con el
		uso de una clase llamada \textit{InputMoleculePDB} que se encargar de realizar 
		la lectura de un archivo en formato \textit{PDB} que contiene datos 
		fundamentales para el programa como: 
		
		\begin{itemize}
			\item Número atómico y número de átomos presentes en la proteína.
			\item Las coordenadas átomicas del sistema expresada en $\textup{\r{A}}$ngstr\"oms.
		\end{itemize}
		
		\item \textbf{Paso 2}: Usa una clase llamada \textit{Molecule}
		que esta diseñanda para representar una molécula como un conjunto de 
		átomos, en este paso \textit{Molecule} se encarga de extraer y guardar las coordenas 
		átomicas del sistema $(x, y, z)$.\\
		
		\item \textbf{Paso 3}: Nuevamente se usa la clase 
		\textit{Molecule} para el cálculo de propiedades moleculares como:
		
		\begin{itemize}
			\item Deterimnar la fórmula empírica de la molécula, contado la cantidad de átomos por número 
			atómico.
			\item Devolver valores como la cantidad de átomos de un tipo específico 
			dado por el número átomico.
			\item Calcular los extremos mínimos y máximos de todos los átomos para definir 
			el radio máximo desde el origen de la esfera que envuelve a la proteína.
			\item Calcula el centro de masa considerando las masas atómicas.
			\item Calcula el centroide (promedio aritmético de las coordenadas de todos los
			átomos, sin considerar su masa).
		\end{itemize}
		
		Además, la clase \textit{Molecule} también se encarga de trasladar la proteína de 
		modo que su centroide coincida con el origen de las coordenas $(0, 0, 0)$.  
		
		
		\item \textbf{Paso 4}: \textit{Molecule} también se encarga de identificar enlaces 
		covalentes entre átomos según la distancia que exista entre ellos y 
		sus radios de Van Der Waals para devolver una lista con los índices de vecinos 
		cercanos a un átomo dado.
		
	\end{itemize}

 
 
 
 
 
 
 
 
 
 
 
 
 

El propósito principal del programa \textit{molmassfractaldim} es determinar la dimesi\'{o}n fractal de masa mediante la caracterizaci\'{o}n de la geometría interna de un sistema proteico. Esto se logra construyendo una tabla de \(r_i\), \( \bar{N}(r_i)\) y \( \bar{M}(r_i)\), donde;

\begin{itemize}
	\item \(r_i\) es el radio de medida centrado en posiciones aleatorias de una prote\'{i}na con valores m\'{i}nimos y m\'{a}ximos definidos como \(mr_{min}\) y \(mr_{max}\).
	\item \( \bar{N}(r_i)\) es el número promedio de partículas contenidas dentro de un radio \(r_i\). 
	\item  \( \bar{M}(r_i)\) es el n\'{u}mero de masa promedio de las part\'{i}culas contenidas en el radio \(r_i\).
\end{itemize}








Posteriormente, los valores de \( \bar{N}(r_i) \) y \( \bar{M}(r_i)\) pueden analizarse mediante regresiones lineales para determinar la dimensi\'{o}n fractal de masa. Para calcular los valores de \(r_i\), \( \bar{N}(r_i)\) y \( \bar{M}(r_i)\), el algoritmo sigue los siguientes pasos:


\begin{enumerate}
	\item Inicializaci\'{o}n\\
	 Se carga un objeto llamado Molecule que contiene la informaci\'{o}n estructural del sistema (coordenadas y masas de los elementos presentes en la prote\'{i}na).
	
	\item Configuraci\'{o}n de par\'{a}metros\\
	Se calcula el centro de masas del cl\'{u}ster.\\
	Se recentran las part\'{i}culas de la prote\'{i}na en el origen del sistema de coordenadas.\\
	Se define el radio m\'{a}ximo del cluster \(R_{max}\) (la mayor distancia entre un \'{a}tomo y el origen).\\
	Se define el radio m\'{a}ximo de b\'{u}squeda de semillas como \(seekR_{max}\) para evitar que los c\'{i}rculos de medici\'{o}n se salgan del cl\'{u}ster.\\
	Se definen los radios m\'{i}nimo y m\'{a}ximo de medida como \(mr_{min}\) y \(mr_{max}\).\\
	Se calcula el n\'{u}mero de divisiones como \(nr\) y el incremento radial \(dr\).\\
	Se define el n\'{u}mero de medidas por centro como nMeas y el n\'{u}mero total de c\'{i}rculos a considerar.
	
	\item Generaci\'{o}n de puntos de medida\\
	Se eligen de manera aleatoria part\'{i}culas del sistema como centros de las circunferencias de medida. Para cada centro, se identifican las part\'{i}culas contenidas dentro del radio m\'{a}ximo (\(R_{max}\)).
	
	\item Acumulaci\'{o}n de datos\\
	Para cada radio \( r_i \), se cuenta cu\'{a}ntas part\'{i}culas est\'{a}n contenidas dentro de ese radio respecto al centro actual. Esta cuenta se repite para m\'{u}ltiples centros y se promedia para obtener \( \bar{N}(r_i) \).
	
	
	\item Salida de resultados\\
	Los vectores \(r_{i}\), \( \bar{N}(r_i) \) y \( \bar{M}(r_i)\) pueden visualizarse en pantalla o guardarse en un archivo externo.
\end{enumerate}




