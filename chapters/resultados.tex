\chapter{Resultados y discusión}

El propósito principal del programa es determinar la dimesi\'{o}n fractal de masa mediante la caracterizaci\'{o}n de la geometría interna de un sistema molecular. Esto se logra construyendo una tabla de \(r_i\), \( \bar{N}(r_i)\) y \( \bar{M}(r_i)\), donde;

\begin{itemize}
	\item \(r_i\) es l radio de medida centrado en posiciones aleatorias de una proteína.
	\item \( \bar{N}(r_i)\) es el número promedio de partículas contenidas dentro de un radio \(r_i\). 
	\item  \( \bar{M}(r_i)\) es l n\'{u}mero de masa promedio de las part\'{i}culas contenidas en el radio \(r_i\).
\end{itemize}

Posteriormente, los valores de \( \bar{N}(r_i) \) y \( \bar{M}(r_i)\) pueden analizarse mediante ajuste logarítmico para determinar la dimensión fractal de masa del sistema. Para calcular los valores de \(r_i\), \( \bar{N}(r_i)\) y \( \bar{M}(r_i)\), el algoritmo sigue los siguientes pasos:

\begin{enumerate}
	\item Inicialización\\
	 Se carga un objeto llamado Molecule que contiene la información estructural del sistema (coordenadas y masas de los elementos prosentes en la estructura molecular).
	
	\item Configuración de parámetros\\
	Se definen los radios mínimo y máximo de medida (mrmin y mrmax).\\
	Se calcula el número de divisiones (nr) y el incremento radial (dr).\\
	Se define el número de medidas por centro (nMeas) y el número total de círculos a considerar.
	
	\item Generación de puntos de medida\\
	Se eligen de manera aleatoria partículas del sistema como centros de las circunferencias de medida.\\
	Para cada centro, se identifican las partículas contenidas dentro del radio máximo mrmax.
	
	\item Acumulación de datos\\
	Para cada radio \( r_i \), se cuenta cuántas partículas están contenidas dentro de ese radio respecto al centro actual.\\
	Esta cuenta se repite para múltiples centros y se promedia para obtener \( \bar{N}(r_i) \).
	
	
	\item Salida de resultados\\
	Los vectores \( r \), \( \bar{N}(r_i) \) y \( \bar{M}(r_i)\) pueden visualizarse en pantalla o guardarse en un archivo externo.
\end{enumerate}
