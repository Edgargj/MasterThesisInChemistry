\chapter{Marco te\'{o}rico}



\section{Ley de potencias} 

Existe una estrecha relación entre la ley de potencia y el estudio de fractales. La ley de potencia es una relación matemática de la siguiente forma:

\begin{equation}
	y(x) = cx^{a}
	\label{eq:3.1}
\end{equation}

Donde:\\
$c$ es una constante.\\
$a$ es exponente de la ley de potencias.\\
$x$ y $y$ son las variables dependientes e independientes. 

La ecuación \ref{eq:3.1} es una función matemática que relaciona dos cantidades, donde el cambio en una cantidad que da como resultado un cambio en la otra cantidad que es proporcional al cambio elevado a un exponente constante. Es decir, una cantidad varía como potencia de otra. El cambio es independiente del tamaño inicial de dichas cantidades \cite{Meakin1998}.



\section{Fractales}

Los fractales son objetos geométricos que presentan una estructura fragmentada y aparentemente irregular que se manifiestan en muchos contextos tanto naturales como los copos de nieve, como artísticos por ejemplo, en las obras de M. C. Escher, así como en fenómenos físicos, \textit{vervi gratia} los anillos de Saturno, entre otras áreas. El término fractal  proviene del latín \textit{fractus} que significa ``roto o quebrado'' y fue acuñada por el matem\'{a}tico polaco B. Mandelbrot, padre de los fractales.

La definición de fractal depende del área en la que se esté trabajando, sin embargo existen varias propiedades que tienen estos objetos y si alguna de ellas se cumple entonces se dice que es un objeto fractal:

\begin{itemize}
	\item Es autosimilar.
	\item Su dimensión de Hausdorff-Besicovich es mayor que su dimensión topológica.
	\item No es diferenciable en ningún punto.
	\item Tiene una longitud o complejidad infinita.
\end{itemize}

\clearpage

La  propiedad mas importante en fractales es la autosimilitud. La autosimilitud implica observar como las mismas propiedades o caracter\'{i}sticas de un objeto se replican a distintas escalas. En fractales matem\'{a}ticos o puros como el conjunto de Mandelbrot, la autosimilitud es exactamente igual sin importar el n\'{u}mero de veces que se amplie, siempre se ver\'{a} lo mismo. A la autosimilitud anterior se le conoce como autosimilitud exacta. Otro tipo de autosimilitud, es la autosimilitud aproximada que normalmente aparece en objetos donde sus partes o conjuntos de esas partes son muy similares (aunque no ind\'{e}nticas) y en su mayor\'{i}a, aparecen en la naturaleza, como en las arterias o venas de la retina de un ojo. A diferencia de la autosimilitud exacta, la autosimilitud aproximada solo se puede replicar una cierta cantidad de veces y despu\'{e}s, se pierde esa autosimilitud.

Dicho todo lo anterior, podríamos decir que un fractal es un objeto geométrico en el que una misma estructura irregular o aparentemente fragmentada se repite a diferentes escalas y tamaños. 

\section{Dimensi\'{o}n fractal}

En geometr\'{i}a fractal existe un concepto fundamental conocido como dimensi\'{o}n fractal, que es una generalizaci\'{o}n del concepto de dimensi\'{o}n en geometr\'{i}a Euclidiana. La dimensi\'{o}n fractal es una medida de cu\'{a}nto aparenta llenar el espacio un objeto conforme aumentamos o disminuimos la escala de an\'{a}lisis.

%B. Mandelbrot utiliz\'{o} la dimensi\'{o}n fractal para calcular la longitud de la costa de Gran Bretaña, a este problema se le conoce como paradoja de la costa. Demostró que al usar reglas de diferentes tamaños para medir detalladamente los contornos de la costa, se obtienen longitudes mayores (tal como se observa en la Figura ), sugiriendo que la costa tiene una longitud infinitamente compleja, contrario a lo que se pensar\'{i}a. 


De la  amplia variedad de dimensiones fractales que existen la definición de la dimensión fractal de Hausdorff-Besicovich es probablemente la más importante porque puede definirse para cualquier conjunto y es matemáticamente práctica, ya que se basa en medidas relativamente faciles de manipular.

Existen diferentes formas de determinar la dimesi\'{o}n fractal, una muy conocida en ciencias de materiales es la dimensi\'{o}n fractal de masa, y es \'{u}til porque permite cuantificar la cantidad masa que hay dentro de la estructura que se desea analizar. La dimensi\'{o}n fractal de masa consiste en elegir un punto o centro al azar dentro de la estructura, trazar un c\'{i}rculo alrededor de ese punto y contar el n\'{u}mero de objetos que hay dentro de ese c\'{i}rculo. Nuevamente, se repite el mismo proceso pero con un nuevo c\'{i}rculo que se encuentra dentro del que primero c\'{i}rculo y se procede a contar el n\'{u}mero de objetos que dentro de \'{e}l.

 Al repetir el proceso anterior, con diferentes centros y c\'{i}rculos, se puede observar en una gr\'{a}fica, como el n\'{u}mero de objetos \textit{versus} el radio de los distintos c\'{i}rculos sigue la caracteristica ley de potencias. Donde la gr\'{a}fica obtenida a partir de ajustes lineales, resulta tener una pendiente igual a D, donde D es la dimensi\'{o}n fractal de masa. 

\section{Multifractalidad}

Otro concepto desarrollado por B. Mandelbrot en su libro \textit{Fractal Geometry of Nature} fue la multifractalidad. La multifractalidad es una propiedad de ciertos sistemas o estructuras complejas que presentan un crecimiento fractal pero de manera heterog\'{e}nea en distintas regiones o escalas. A diferencia de los fractales, que son descritos con una \'{u}nica dimensi\'{o}n fractal (un valor constante que representa la relaci\'{o}n entre el detalle del patr\'{o}n y la escala), en un sistema multifractalidad existen multiples dimensiones fractales que reflejan la variablidad de la distribuci\'{o}n y concentraci\'{o}n de sus elementos. 

En t\'{e}rminos simples, un sistema multifractal est\'{a} compuesto de distintas subestructuras que tienen diferentes grados de ``irregularidad'', lo que significa que puede ser descrito con la una dimensi\'{o}n fractal. Cada regi\'{o}n del sistema podr\'{i}a necesitas una dimensi\'{o}n fractal espec\'{i}fica para describir su complejidad.


\section{Sistemas magnetoreol\'{o}gicos}

Los sistemas magnetoreol\'{o}gicos son fluidos complejos que contienen part\'{i}culas magn\'{e}ticas (\'{o}xidos de metales) suspendidos en un l\'{i}quido (como aceite de silicona). La caracter\'{i}stica principal de estos sistemas es que sus propiedades mec\'{a}nicas cambian dr\'{a}sticamente cuando se aplica un campo magn\'{e}tico externo. En ausencia de este campo, el fluido se comporta como un l\'{i}quido ordinario, pero al aplicar un campo magn\'{e}tico, las part\'{i}culas magn\'{e}ticas se alinean y forman estructuras organizadas (como cadenas o fibras) que transforman el fluido en una especie de gel envegecido o s\'{o}lido semirr\'{i}gido en cuesti\'{o}n de milisegundos. 


\section{Multifractalidad en sistemas magnetoreol\'{o}gicos}

Para observar los diferentes patrones de agregaci\'{o}n en presencia del campo magn\'{e}tico, Carrillo y colaboradores utilizaron bajas concentraciones de part\'{i}culas, de menos de 0.1 en fracci\'{o}n de volumen. Observando diferentes etapas del proceso de agregaci\'{o}n. Y por lo tanto, al determinar la dimensi\'{o}n fractal de masa observaron una variaci\'{o}n de la dimensi\'{o}n fractal que son precisamente las 3 porciones en la gr\'{a}fica que est\'{a}n asociadas a 3 etapas de agregaci\'{o}n.
 
 
 
\section{Niveles de organizaci\'{o}n en la estructura de las prote\'{i}nas}


Las prote\'{i}nas est\'{a}n presentes en todos los sistemas vivos, desde estructuras como la hemoglobina, el tejido cerebral y una cantidad considerable de esas prote\'{i}nas se han cristalizado para
posteriormente, caracterizarse por m\'{e}todos como RMN, R-X, ME. Y una vez hecho lo anterior, los datos son enviados y son revisados por expertos biocuradores, despu\'{e}s de
ser aprobados se ponen a disposici\'{o}n de forma gratuita bajo algún dominio como el \textit{Protein Data Bank}.


Como es bien sabido, las prote\'{i}nas son pol\'{i}meros lineales compuestos por amino\'{a}cidos y aunque en la
c\'{e}lula existen m\'{a}s de 60 amino\'{a}cidos distintos, solo 20 de ellos son incorporados en la mayor\'{i}a de las
prote\'{i}nas.
Cada amino\'{a}cido tiene una estructura b\'{a}sica que es un grupo carboxilo, un grupo amino, un \'{a}tomo de
hidr\'{o}geno y un grupo r todos ellos unidos a un \'{a}tomo de carbono quiral. A esta uni\'{o}n de todos los grupos
mencionados se le conoce como enlace pept\'{i}dico.

Podemos describir a las prote\'{i}nas en cuatro niveles jer\'{a}rquicos:

\begin{itemize}

\item La estructura primaria donde se encuentra la secuencia de los
amino\'{a}cidos del polip\'{e}ptido. Cuando se describe la estructura primaria de una prote\'{i}na se especifica el
orden en el que aparecen los amino\'{a}cidos desde un extremo de la mol\'{e}cula hasta el otro extremo.

\item La estructura secundaria son patrones repetitivos de una estructura local, en estos
patrones se presentan la formaci\'{o}n de puentes de hidr\'{o}geno y estas interacciones locales son las
responsables de las 2 principales conformaciones secundarias de la prote\'{i}na que son denominadas alfa
h\'{e}lice y lamina beta.


\item La estructura terciaria \'{e}sta basada en interacciones entre grupos laterales que tienen diferentes
propiedades: como los grupos hidr\'{o}fobos y amino\'{a}cidos polares. Como resultado de esto, la cadena
polipept\'{i}dica se pliega, se enrolla y gira en la conformaci\'{o}n nativa que est\'{a} elija. Que normalmente es la
conformaci\'{o}n m\'{a}s estable para una determinada secuencia de amino\'{a}cidos.


\item Por último, la estructura cuaternaria es el nivel de organizaci\'{o}n que concierna las
interacciones y el ensamblaje de las subunidades proteicas, esta categor\'{i}a incluye muchas prote\'{i}nas
principalmente aquellas cuyo peso molecular supera las 50, 000 unidades.

\end{itemize}

\section{Medidas para comparar estructuras proteicas}

Para evaluar y comparar estructuras proteicas a los niveles mencionados anteriormente se utilizan
diferentes medidas cuantitativas y cualitativas. Aunque el problema parece sencillo su cuantificaci\'{o}n es
compleja y sigue evolucionando. Una medida usada por los qu\'{i}micos computacionales es el RMSD. El acrónimo RMSD quiere decir desviación
cuadrática media de un conjunto de coordenadas atómicas. Y la gr\'{a}fica que se observa en la figura 
es producida mediante dinámica molecular y en ella, se estudia cómo evolucionan las posiciones de los
átomos de la proteína que se analiza a través del tiempo. El RMSD se calcula con base en una estructura
de referencia que generalmente es la posición inicial en el tiempo cero de la simulación.
Es importante señalar que, esta es una medida de toda la proteína, en la gr\'{a}fica no analiza si la estructura
proteica se está deshaciendo o se est\'{a} expandiendo o se está torciendo. Solo es, c\'{a}nto en promedio se desvían los \'{a}tomos de su posición original. Por lo tanto, encontrar
m\'{e}todos alternativos al RMSD o encontrar enfoques complementarios para comparar estructuras
sigue siendo un reto para la comunidad cient\'{i}fica.
