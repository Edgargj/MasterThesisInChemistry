\chapter{Marco te\'{o}rico}

\section{Fractales}

Los fractales son objetos geom\'{e}tricos investigados en los años setenta por el matem\'{a}tico polaco B.  Mandelbrot. La principal caracter\'{i}stica de los fractales es la autosimilitud o autosemejanza. En palabras simples, la autosimilitud implica observar como las mismas propiedades o caracter\'{i}sticas del objeto que se observa se replican a distintas escalas.

Cuando se hace una ampliaci\'{o}n en una imagen como  podr\'{i}amos ver que esa parte se ampli\'{o}, se replica a escalas mayores y precisamente en fractales matem\'{a}ticos o puros, \textit{vervi gratia} conjunto de Mandelbrot, es exactamente igual sin importar el n\'{u}mero de veces que se amplie, siempre se ver\'{a} lo mismo. A la autosimilitud anterior se le conoce como autosimilitud exacta.

Otro tipo de autosimilitud es la autosimilitud aproximada y normalmente aparece en objetos geom\'{e}tricos donde sus partes o conjuntos de esas partes son muy similares (aunque no ind\'{e}nticas) y en su mayor\'{i}a, aparecen en la naturaleza, v\'{e}ase la figura donde solo se puede replicar la autosimilitud una cierta cantidad de veces y despu\'{e}s, se pierde esa autosimilitud.

Dicho lo anterior, un fractal es un objeto geom\'{e}trico en que el una misma estructura fragmentada o aparentemente irregular se repite a diferentes escalas y tamaños. 

\section{Dimensi\'{o}n fractal}

En geometr\'{i}a fractal existe un concepto fundamental conocido como dimens\'{o}n fractal, que es una generalizaci\'{o}n del concepto de dimensi\'{o}n en geometr\'{i}a cl\'{a}sica. Por lo tanto, podr\'{i}amos decir que la dimensi\'{o}n fractal es una medida de cu\'{a}nto aparenta llenar el espacio un objeto conforme amentamos o disminuimos la escala de an\'{a}lisis.

El padre de los fractales, B. Mandelbrot utiliz\'{o} la dimensi\'{o}n fractal para calcular la longitud de la costa de Inglaterra. Us\'{o} reglas de diferentes tamaños para medir detalladamente los contornos de la costa, a medida que utilizaba reglas m\'{a}s pequeñas, obten\'{i}a longitudes mayores, esto sugiri\'{o} que la costa tiene una longitud infinita, contrario a lo que se pensar\'{i}a. 

En este contexto, existen multiples formas de determinar la dimesi\'{o}n fractal, sin embargo, una especialmente \'{u}til es la dimensi\'{o}n de fractal de masa, porque permite cuantificar la cantidad masa que hay dentro una estructura. \'{E}sta, consiste en eligir un punto o centro al azar dentro de la estructura que se desea analizar, trazar un c\'{i}rculo alrededor de ese centro y contar el n\'{u}mero de part\'{i}culas que hay dentro, despu\'{e}s, se repite el proceso de buscar un centro al azar y variar el radio del nuevo c\'{i}rculo contado el n\'{u}mero de part\'{i}culas que hay en el nuevo centro. Si se repite este proceso con diferentes centros y c\'{i}rculos, se puede graficar el n\'{u}mero de part\'{i}culas promedio \textit{versus} los radios de los c\'{i}rculos que se han definido. Hecho lo anterior, se obtendr\'{i}a una gr\'{a}fica como la figura que resulta tener una pediente igual a D, donde D es la dimensi\'{o}n fractal de masa. 

\section{Multifractalidad}

La multifractalidad es una propiedad de ciertos sistemas o estructuras complejas que presentan un crecimiento fractal pero de manera heterog\'{e}nea en distintas regiones o escalas. A diferencia de los fractales, que son descritos con una \'{u}nica dimensi\'{o}n fractal (un valor constante que representa la relaci\'{o}n entre el detalle del patr\'{o}n y la escala), en un sistema multifractalidad existen multiples dimensiones fractales que reflejan la variablidad de la distribuci\'{o}n y concentraci\'{o}n de sus elementos. 

En t\'{e}rminos simples, un sistema multifractal est\'{a} compuesto de distintas subestructuras que tienen diferentes grados de ``irregularidad'', lo que significa que puede ser descrito con la una dimensi\'{o}n fractal. Cada regi\'{o}n del sistema podr\'{i}a necesitas una dimensi\'{o}n fractal espec\'{i}fica para describir su complejidad.


\section{Sistemas magnetoreol\'{o}gicos}

Los sistemas magnetoreol\'{o}gicos son fluidos complejos que contienen part\'{i}culas magn\'{e}ticas (\'{o}xidos de metales) suspendidos en un l\'{i}quido (como aceite de silicona). La caracter\'{i}stica principal de estos sistemas es que sus propiedades mec\'{a}nicas cambian dr\'{a}sticamente cuando se aplica un campo magn\'{e}tico externo. En ausencia de este campo, el fluido se comporta como un l\'{i}quido ordinario, pero al aplicar un campo magn\'{e}tico, las part\'{i}culas magn\'{e}ticas se alinean y forman estructuras organizadas (como cadenas o fibras) que transforman el fluido en una especie de gel envegecido o s\'{o}lido semirr\'{i}gido en cuesti\'{o}n de milisegundos. 


\section{Multifractalidad en sistemas magnetoreol\'{o}gicos}

Para observar los diferentes patrones de agregaci\'{o}n en presencia del campo magn\'{e}tico, Carrillo y colaboradores utilizaron bajas concentraciones de part\'{i}culas, de menos de 0.1 en fracci\'{o}n de volumen. Observando diferentes etapas del proceso de agregaci\'{o}n. Y por lo tanto, al determinar la dimensi\'{o}n fractal de masa observaron una variaci\'{o}n de la dimensi\'{o}n fractal que son precisamente las 3 porciones en la gr\'{a}fica que est\'{a}n asociadas a 3 etapas de agregaci\'{o}n.
 
 
 
\section{Niveles de organizaci\'{o}n en la estructura de las prote\'{i}nas}


Las prote\'{i}nas est\'{a}n presentes en todos los sistemas vivos, desde estructuras como la hemoglobina, el tejido cerebral y una cantidad considerable de esas prote\'{i}nas se han cristalizado para
posteriormente, caracterizarse por m\'{e}todos como RMN, R-X, ME. Y una vez hecho lo anterior, los datos son enviados y son revisados por expertos biocuradores, despu\'{e}s de
ser aprobados se ponen a disposici\'{o}n de forma gratuita bajo algún dominio como el \textit{Protein Data Bank}.


Como es bien sabido, las prote\'{i}nas son pol\'{i}meros lineales compuestos por amino\'{a}cidos y aunque en la
c\'{e}lula existen m\'{a}s de 60 amino\'{a}cidos distintos, solo 20 de ellos son incorporados en la mayor\'{i}a de las
prote\'{i}nas.
Cada amino\'{a}cido tiene una estructura b\'{a}sica que es un grupo carboxilo, un grupo amino, un \'{a}tomo de
hidr\'{o}geno y un grupo r todos ellos unidos a un \'{a}tomo de carbono quiral. A esta uni\'{o}n de todos los grupos
mencionados se le conoce como enlace pept\'{i}dico.

Podemos describir a las prote\'{i}nas en cuatro niveles jer\'{a}rquicos:

\begin{itemize}

\item La estructura primaria donde se encuentra la secuencia de los
amino\'{a}cidos del polip\'{e}ptido. Cuando se describe la estructura primaria de una prote\'{i}na se especifica el
orden en el que aparecen los amino\'{a}cidos desde un extremo de la mol\'{e}cula hasta el otro extremo.

\item La estructura secundaria son patrones repetitivos de una estructura local, en estos
patrones se presentan la formaci\'{o}n de puentes de hidr\'{o}geno y estas interacciones locales son las
responsables de las 2 principales conformaciones secundarias de la prote\'{i}na que son denominadas alfa
h\'{e}lice y lamina beta.


\item La estructura terciaria \'{e}sta basada en interacciones entre grupos laterales que tienen diferentes
propiedades: como los grupos hidr\'{o}fobos y amino\'{a}cidos polares. Como resultado de esto, la cadena
polipept\'{i}dica se pliega, se enrolla y gira en la conformaci\'{o}n nativa que est\'{a} elija. Que normalmente es la
conformaci\'{o}n m\'{a}s estable para una determinada secuencia de amino\'{a}cidos.


\item Por último, la estructura cuaternaria es el nivel de organizaci\'{o}n que concierna las
interacciones y el ensamblaje de las subunidades proteicas, esta categor\'{i}a incluye muchas prote\'{i}nas
principalmente aquellas cuyo peso molecular supera las 50, 000 unidades.

\end{itemize}

\section{Medidas para comparar estructuras proteicas}

Para evaluar y comparar estructuras proteicas a los niveles mencionados anteriormente se utilizan
diferentes medidas cuantitativas y cualitativas. Aunque el problema parece sencillo su cuantificaci\'{o}n es
compleja y sigue evolucionando. Una medida usada por los qu\'{i}micos computacionales es el RMSD. El acrónimo RMSD quiere decir desviación
cuadrática media de un conjunto de coordenadas atómicas. Y la gr\'{a}fica que se observa en la figura 
es producida mediante dinámica molecular y en ella, se estudia cómo evolucionan las posiciones de los
átomos de la proteína que se analiza a través del tiempo. El RMSD se calcula con base en una estructura
de referencia que generalmente es la posición inicial en el tiempo cero de la simulación.
Es importante señalar que, esta es una medida de toda la proteína, en la gr\'{a}fica no analiza si la estructura
proteica se está deshaciendo o se est\'{a} expandiendo o se está torciendo. Solo es, c\'{a}nto en promedio se desvían los \'{a}tomos de su posición original. Por lo tanto, encontrar
m\'{e}todos alternativos al RMSD o encontrar enfoques complementarios para comparar estructuras
sigue siendo un reto para la comunidad cient\'{i}fica.