\chapter{Funcionamiento del programa}
\label{chap:funcmolmass}


El prop\'{o}sito principal del programa \textit{molmassfractaldim} es determinar la dimensi\'{o}n fractal de masa mediante la caracterizaci\'{o}n de la geometr\'{i}a interna de un sistema proteico. Esto se logra construyendo una tabla de $r_{k}$, $\langle N(r_k) \rangle$, $\langle M(r_k) \rangle$ y $\langle Rg(r_k) \rangle$ donde:

\begin{itemize}
	\item $r_{k}$ es el radio de medida centrado en posiciones aleatorias de una prote\'{i}na con valores m\'{i}nimos y m\'{a}ximos definidos como $r_{min}$ y $r_{max}$.
	\item $\langle N(r_k) \rangle$ es el n\'{u}mero promedio de part\'{i}culas contenidas dentro de un radio $r_{k}$. 
	\item  $\langle M(r_k) \rangle$ es el n\'{u}mero de masa promedio de las part\'{i}culas contenidas en el radio $r_{k}$.
	\item  $\langle Rg(r_k) \rangle$ es el radio de giro promedio de las part\'{i}culas contenidas en el radio $r_{k}$.
\end{itemize}

Posteriormente, los valores de $\langle N(r_k) \rangle$, $\langle M(r_k) \rangle$ y $\langle Rg(r_k) \rangle$ pueden analizarse mediante regresiones lineales para determinar la dimensi\'{o}n fractal de masa.


	\clearpage

\subsection*{Banderas de entrada}

	El programa \emph{molmassfractaldim}, incorpora banderas de entrada al c\'{o}digo para tener distintas opciones en el c\'{a}lculo de la dimensi\'{o}n fractal de masa. Las banderas incorporadas se muestran en la Tabla \ref{tab:opciones_programa}.
	
	\begin{table}[H]
		\centering
		\begin{tabular}{lp{11cm}}
			\hline
			\textbf{Opci\'{o}n} & \textbf{Descripci\'{o}n} \\ \hline
			-a A & Establece el n\'{u}mero de medidas a realizar. Esto se refiere a la cantidad de veces que el programa calcular\'{a} el promedio de $\langle N(r_k) \rangle$ $\langle M(r_k) \rangle$ y $\langle R_g(r_k) \rangle$. Es \'{u}til si se desea promediar los resultados sobre diferentes configuraciones o pruebas. Para m\'{a}s detalles sobre esta opci\'{o}n, v\'{e}ase la secci\'{o}n \ref{subsec:detM(r)}. 
			\\ 
			-n k & Define el n\'{u}mero de radios $r_{k}$ que se utilizar\'{a}n. Es decir, el n\'{u}mero de filas en la tabla de $r_{k}$, $\langle N(r_k) \rangle$, $\langle M(r_k) \rangle$ y $\langle Rg(r_k) \rangle$.
			\\
			-l & Permite espaciar los radios multiplicativamente. Es decir, los radios de medici\'{o}n est\'{a}n igualmente separados en una escala logar\'{i}tmica, en lugar de una escala lineal.
			\\
			-q & Modo silencioso. Al activar esta opci\'{o}n, la mayor\'{i}a de los mensajes informativos que normalmente se mostrar\'{i}an en la salida est\'{a}ndar se suprimir\'{a}n. 
			\\ 
			-r rmn & Establece el radio m\'{i}nimo $r_{min}$ para las mediciones. Esto es \'{u}til para limitar el rango de los radios de las esferas que se tomar\'{a}n en cuenta en los c\'{a}lculos; $r$ se refiere al radio de la esfera. \\ 
			-R rmx & Establece el radio m\'{a}ximo $r_{max}$ para las mediciones. Similar a la opci\'{o}n anterior, permite definir el rango m\'{a}ximo de los radios de las esferas. 
			\\ 
			-V & Muestra la versi\'{o}n del programa. 
			\\ 
			-h & Muestra el men\'{u} de ayuda con la descripci\'{o}n de las opciones y el uso del programa. 
			\\ \hline
		\end{tabular}
		\caption{Banderas de entrada para el programa \emph{molmassfractaldim}.}
		\label{tab:opciones_programa}
	\end{table}


\chapter{Conjunto de 9 prote\'{i}nas}
\label{chap:9P}

El an\'alisis global del conjunto proteico (9 prote\'{i}nas) presentado en la secci\'on \ref{subsec:pfp}, demuestran un comportamiento multifractal en toda la muestra. Dado que este patr\'on se manifiesta de manera consistente (aunque con una intensidad variable) en todos los sistemas, el an\'alisis individual no aportan informaci\'on adicional para verificar la existencia del fen\'omeno. Por consiguiente, se ha decidido ubicar los resultados en este ap\'endice (junto con una imagen de cada sistema), donde el lector podr\'a examinar los detalles particulares de cada sistema. Para comodidad del lector, nuevamente se expone la Tabla \ref{Tabla:ids9}.

\begin{table}[h!]
	\centering
		\begin{footnotesize}
			\begin{tabular}{||lllc||}
				\multicolumn{4}{l}{} \\ 
				\hline
				ldx & IdPDB & Nombre de prote\'{i}na & N\'{u}mero de \'{a}tomos \\
				\hline
				1  & 1a2b & Transformadora RHOA & 2831 \\
				2 & 1b3e & Transferrina s\'{e}rica & 5037 \\
				3 & 11gs & Glutati\'{o}n s-transferasa & 6536 \\ 
				4 & 1auk & Arilsulfatasa A & 7086 \\
				5 & 1a8m & Factor de necr\'{o}sis tumoral $\alpha$ & 7104 \\
				6 & 1a52 & Receptor de estr\'{o}geno & 7765 \\
				7 & 1a3n & Hemoglobina humana desoxidante cadena $\alpha$ & 8734 \\
				8 & 1a9w & Hemoglobina embrionaria cadena $\alpha$ & 8820 \\
				9 & 7khw & Translocon EspA & 131200 \\
				\hline
			\end{tabular}
		\end{footnotesize}
	\caption{Nuevo \'{i}ndice de prote\'{i}nas (ldx), Identificadores del \emph{Protein Data Bank} (PDB), nombre de la prote\'{i}na y n\'{u}mero de \'{a}tomos presentes en cada estructura.}
\end{table}


	\begin{figure}[H]
	\subsection*{Prote\'{i}na Transformadora RHOA (IdPDB:1a2b)}
		
	\begin{subfigure}{0.49\textwidth}
		\centering
	\begin{overpic}[width=\linewidth]{graphs/PDBs/1a2b/1a2baddH.pdf}
		\put(-18,32){\includegraphics[width=0.9\linewidth]{graphs/PDBs/1a2b/1a2baddH.png}}
		\end{overpic}
		\caption{(1)}
	\end{subfigure}
	\hspace{0.2cm}
	\begin{subfigure}{0.49\textwidth}
		\centering
		\includegraphics[width=\linewidth,page=1]{graphs/PDBs/1a2b/1a2bEm.pdf}
		\caption{(2)}
	\end{subfigure}
	
	\vspace{0cm} % Espacio entre filas
	
	\hspace{-0.3cm} 
	\begin{subfigure}{0.49\textwidth}
		\centering
		\includegraphics[width=\linewidth,page=1]{graphs/PDBs/1a2b/1a2bEq.pdf}
		\caption{(3)}
	\end{subfigure}
	\hspace{0.2cm}
	\begin{subfigure}{0.49\textwidth} % M\'{a}s ancho para centrar
		\centering
		\includegraphics[width=\linewidth,page=1]{graphs/PDBs/1a2b/1a2b1ns.pdf}
		\caption{(4)}
	\end{subfigure}
	
	\caption{
		Regresiones lineales de $\log_{10}r$ vs $\log_{10}M(r)$ correspondiente a cuatro etapas de procesamiento de la primera prote\'{i}na con \textit{IdPDB:1a2b} de la Tabla \ref{Tabla:ids9}: (1) Adici\'{o}n de \'{a}tomos de hidr\'{o}geno al sistema proteico; (2) al minimizar la energ\'{i´}a de la estructura molecular; (3) equilibrando el sistema bajo condiciones termodin\'{a}micas controladas; y (4) despu\'{e}s de una din\'{a}mica molecular de 1 ns.}
	\label{fig:1a2b}
\end{figure}

\begin{figure}[H]
	\subsection*{Prote\'{i}na Transferrina s\'{e}rica (IdPDB:1b3e)}
	\hspace{-0.3cm} 
	\begin{subfigure}{0.49\textwidth}
		\centering
		\begin{overpic}[width=\linewidth]{graphs/PDBs/1b3e/1b3eaddH.pdf}
			\put(-12,32){\includegraphics[width=0.8\linewidth]{graphs/PDBs/1b3e/1b3eaddH.png}}
		\end{overpic}
		\caption{(1)}
	\end{subfigure}
	\hspace{0.2cm}
	\begin{subfigure}{0.49\textwidth}
		\centering
		\includegraphics[width=\linewidth,page=1]{graphs/PDBs/1b3e/1b3eEm.pdf}
		\caption{(2)}
	\end{subfigure}
	
	\vspace{0cm} % Espacio entre filas
	
	\hspace{-0.3cm} 
	\begin{subfigure}{0.49\textwidth}
		\centering
		\includegraphics[width=\linewidth,page=1]{graphs/PDBs/1b3e/1b3eEq.pdf}
		\caption{(3)}
	\end{subfigure}
	\hspace{0.2cm}
	\begin{subfigure}{0.49\textwidth} % M\'{a}s ancho para centrar
		\centering
		\includegraphics[width=\linewidth,page=1]{graphs/PDBs/1b3e/1b3e1ns.pdf}
		\caption{(4)}
	\end{subfigure}
	\caption{Regresiones lineales de $\log_{10}r$ vs $\log_{10}M(r)$ correspondiente a cuatro etapas de procesamiento de la s\'{e}ptima prote\'{i}na con \textit{IdPDB:1b3e} de la Tabla \ref{Tabla:ids9}: (1) Adici\'{o}n de \'{a}tomos de hidr\'{o}geno al sistema proteico; (2) al minimizar la energ\'{i´}a de la estructura molecular; (3) equilibrando el sistema bajo condiciones termodin\'{a}micas controladas; y (4) despu\'{e}s de una din\'{a}mica molecular de 1 ns.}
	\label{fig:1b3e}
\end{figure}


\begin{figure}[H]
	\subsection*{Prote\'{i}na Glutati\'{o}n s-transferasa (IdPDB:11gs)}
	
	\hspace{-0.3cm} 
	\begin{subfigure}{0.49\textwidth}
		\centering
		\begin{overpic}[width=\linewidth]{graphs/PDBs/11gs/11gsaddH.pdf}
			\put(-17,31){\includegraphics[width=0.85\linewidth]{graphs/PDBs/11gs/11gsaddH.png}}
		\end{overpic}
		\caption{(1)}
	\end{subfigure}
	\hspace{0.2cm}
	\begin{subfigure}{0.49\textwidth}
		\centering
		\includegraphics[width=\linewidth,page=1]{graphs/PDBs/11gs/11gsEm.pdf}
		\caption{(2)}
	\end{subfigure}
	
	\vspace{0cm} % Espacio entre filas
	
	\hspace{-0.3cm} 
	\begin{subfigure}{0.49\textwidth}
		\centering
		\includegraphics[width=\linewidth,page=1]{graphs/PDBs/11gs/11gsEq.pdf}
		\caption{(3)}
	\end{subfigure}
	\hspace{0.2cm}
	\begin{subfigure}{0.49\textwidth} % M\'{a}s ancho para centrar
		\centering
		\includegraphics[width=\linewidth,page=1]{graphs/PDBs/11gs/11gs1ns.pdf}
		\caption{(4)}
	\end{subfigure}
	
	\caption{Regresiones lineales de $\log_{10}r$ vs $\log_{10}M(r)$ correspondiente a cuatro etapas de procesamiento de la octava prote\'{i}na con \textit{IdPDB:11gs} de la Tabla \ref{Tabla:ids9}: (1) Adici\'{o}n de \'{a}tomos de hidr\'{o}geno al sistema proteico; (2) al minimizar la energ\'{i´}a de la estructura molecular; (3) equilibrando el sistema bajo condiciones termodin\'{a}micas controladas; y (4) despu\'{e}s de una din\'{a}mica molecular de 1 ns.}
	\label{fig:11gs}
\end{figure}

\begin{figure}[H]
	\subsection*{Prote\'{i}na Arilsulfatasa A (IdPDB:1auk)}
	\hspace{-0.3cm} 
	\begin{subfigure}{0.49\textwidth}
		\centering
		\begin{overpic}[width=\linewidth]{graphs/PDBs/1auk/1aukaddH.pdf}
			\put(-11,35){\includegraphics[width=0.8\linewidth]{graphs/PDBs/1auk/1aukaddH.png}}
		\end{overpic}
		\caption{(1)}
	\end{subfigure}
	\hspace{0.2cm}
	\begin{subfigure}{0.49\textwidth}
		\centering
		\includegraphics[width=\linewidth,page=1]{graphs/PDBs/1auk/1aukEm.pdf}
		\caption{(2)}
	\end{subfigure}
	
	\vspace{0cm} % Espacio entre filas
	
	\hspace{-0.3cm} 
	\begin{subfigure}{0.49\textwidth}
		\centering
		\includegraphics[width=\linewidth,page=1]{graphs/PDBs/1auk/1aukEq.pdf}
		\caption{(3)}
	\end{subfigure}
	\hspace{0.2cm}
	\begin{subfigure}{0.49\textwidth} % M\'{a}s ancho para centrar
		\centering
		\includegraphics[width=\linewidth,page=1]{graphs/PDBs/1auk/1auk1ns.pdf}
		\caption{(4)}
	\end{subfigure}
	\caption{Regresiones lineales de $\log_{10}r$ vs $\log_{10}M(r)$ correspondiente a cuatro etapas de procesamiento de la sexta prote\'{i}na con \textit{IdPDB:1auk} de la Tabla \ref{Tabla:ids9}: (1) Adici\'{o}n de \'{a}tomos de hidr\'{o}geno al sistema proteico; (2) al minimizar la energ\'{i´}a de la estructura molecular; (3) equilibrando el sistema bajo condiciones termodin\'{a}micas controladas; y (4) despu\'{e}s de una din\'{a}mica molecular de 1 ns.}
	\label{fig:1auk}
\end{figure}





\begin{figure}[H]
	\subsection*{Prote\'{i}na Factor de necr\'{o}sis tumoral $\alpha$ (IdPDB:1a8m)}
	\hspace{-0.3cm} 
	\begin{subfigure}{0.49\textwidth}
		\centering
		\begin{overpic}[width=\linewidth]{graphs/PDBs/1a8m/1a8maddH.pdf}
			\put(-12,34){\includegraphics[width=0.8\linewidth]{graphs/PDBs/1a8m/1a8maddH.png}}
			\end{overpic}
		\caption{(1)}
	\end{subfigure}
	\hspace{0.2cm}
	\begin{subfigure}{0.49\textwidth}
		\centering
		\includegraphics[width=\linewidth,page=1]{graphs/PDBs/1a8m/1a8mEm.pdf}
		\caption{(2)}
	\end{subfigure}
	
	\vspace{0cm} % Espacio entre filas
	
	\hspace{-0.3cm} 
	\begin{subfigure}{0.49\textwidth}
		\centering
		\includegraphics[width=\linewidth,page=1]{graphs/PDBs/1a8m/1a8mEq.pdf}
		\caption{(3)}
	\end{subfigure}
	\hspace{0.2cm}
	\begin{subfigure}{0.49\textwidth} % M\'{a}s ancho para centrar
		\centering
		\includegraphics[width=\linewidth,page=1]{graphs/PDBs/1a8m/1a8m1ns.pdf}
		\caption{(4)}
	\end{subfigure}
	\caption{Regresiones lineales de $\log_{10}r$ vs $\log_{10}M(r)$ correspondiente a cuatro etapas de procesamiento de la cuarta prote\'{i}na con \textit{IdPDB:1a8m} de la Tabla \ref{Tabla:ids9}: (1) Adici\'{o}n de \'{a}tomos de hidr\'{o}geno al sistema proteico; (2) al minimizar la energ\'{i´}a de la estructura molecular; (3) equilibrando el sistema bajo condiciones termodin\'{a}micas controladas; y (4) despu\'{e}s de una din\'{a}mica molecular de 1 ns.}
	\label{fig:1a8m}
\end{figure}

\begin{figure}[H]
	\subsection*{Prote\'{i}na Receptor de estr\'{o}geno (IdPDB:1a52)}
	
	\hspace{-0.3cm} 
	\begin{subfigure}{0.49\textwidth}
		\centering
		\begin{overpic}[width=\linewidth]{graphs/PDBs/1a52/1a52addH.pdf}
		\put(-14,32){\includegraphics[width=0.79\linewidth]{graphs/PDBs/1a52/1a52addH.png}}
			\end{overpic}
		\caption{(1)}
	\end{subfigure}
	\hspace{0.2cm}
	\begin{subfigure}{0.49\textwidth}
		\centering
		\includegraphics[width=\linewidth,page=1]{graphs/PDBs/1a52/1a52Em.pdf}
		\caption{(2)}
	\end{subfigure}
	
	\vspace{0cm} % Espacio entre filas
	
	\hspace{-0.3cm} 
	\begin{subfigure}{0.49\textwidth}
		\centering
		\includegraphics[width=\linewidth,page=1]{graphs/PDBs/1a52/1a52Eq.pdf}
		\caption{(3)}
	\end{subfigure}
	\hspace{0.2cm}
	\begin{subfigure}{0.49\textwidth} % M\'{a}s ancho para centrar
		\centering
		\includegraphics[width=\linewidth,page=1]{graphs/PDBs/1a52/1a521ns.pdf}
		\caption{(4)}
	\end{subfigure}
	\caption{Regresiones lineales de $\log_{10}r$ vs $\log_{10}M(r)$ correspondiente a cuatro etapas de procesamiento de la tercera prote\'{i}na con \textit{IdPDB:1a52} de la Tabla \ref{Tabla:ids9}: (1) Adici\'{o}n de \'{a}tomos de hidr\'{o}geno al sistema proteico; (2) al minimizar la energ\'{i´}a de la estructura molecular; (3) equilibrando el sistema bajo condiciones termodin\'{a}micas controladas; y (4) despu\'{e}s de una din\'{a}mica molecular de 1 ns.}
	\label{fig:1a52}
\end{figure}


\begin{figure}[H]
	\subsection*{Prote\'{i}na Hemoglobina humana desoxidante cadena $\alpha$ (IdPDB:1a3n)}
	
	\hspace{-0.3cm} 
	\begin{subfigure}{0.49\textwidth}
		\centering
		\begin{overpic}[width=\linewidth]{graphs/PDBs/1a3n/1a3naddH.pdf}
		\put(-5,36){\includegraphics[width=0.71\linewidth]{graphs/PDBs/1a3n/1a3naddH.png}}
			\end{overpic}
		\caption{(1)}
	\end{subfigure}
	\hspace{0.2cm}
	\begin{subfigure}{0.49\textwidth}
		\centering
		\includegraphics[width=\linewidth,page=1]{graphs/PDBs/1a3n/1a3nEm.pdf}
		\caption{(2)}
	\end{subfigure}
	
	\vspace{0cm} % Espacio entre filas
	
	\hspace{-0.3cm} 
	\begin{subfigure}{0.49\textwidth}
		\centering
		\includegraphics[width=\linewidth,page=1]{graphs/PDBs/1a3n/1a3nEq.pdf}
		\caption{(3)}
	\end{subfigure}
	\hspace{0.2cm}
	\begin{subfigure}{0.49\textwidth} % M\'{a}s ancho para centrar
		\centering
		\includegraphics[width=\linewidth,page=1]{graphs/PDBs/1a3n/1a3n1ns.pdf}
		\caption{(4)}
	\end{subfigure}
	\caption{Regresiones lineales de $\log_{10}r$ vs $\log_{10}M(r)$ correspondiente a cuatro etapas de procesamiento de la segunda prote\'{i}na con \textit{IdPDB:1a3n} de la Tabla \ref{Tabla:ids9}: (1) Adici\'{o}n de \'{a}tomos de hidr\'{o}geno al sistema proteico; (2) al minimizar la energ\'{i´}a de la estructura molecular; (3) equilibrando el sistema bajo condiciones termodin\'{a}micas controladas; y (4) despu\'{e}s de una din\'{a}mica molecular de 1 ns.}
	\label{fig:1a3n}
\end{figure}
		


\begin{figure}[H]
	\subsection*{Prote\'{i}na Hemoglobina embrionaria cadena $\alpha$ (IdPDB:1a9w)}
	
	\hspace{-0.3cm} 
	\begin{subfigure}{0.49\textwidth}
		\centering
		\begin{overpic}[width=\linewidth]{graphs/PDBs/1a9w/1a9waddH.pdf}
		\put(-10,32){\includegraphics[width=0.8\linewidth]{graphs/PDBs/1a9w/1a9waddH.png}}
			\end{overpic}
		\caption{(1)}
	\end{subfigure}
	\hspace{0.2cm}
	\begin{subfigure}{0.49\textwidth}
		\centering
		\includegraphics[width=\linewidth,page=1]{graphs/PDBs/1a9w/1a9wEm.pdf}
		\caption{(2)}
	\end{subfigure}
	
	\vspace{0cm} % Espacio entre filas
	
	\hspace{-0.3cm} 
	\begin{subfigure}{0.49\textwidth}
		\centering
		\includegraphics[width=\linewidth,page=1]{graphs/PDBs/1a9w/1a9wEq.pdf}
		\caption{(3)}
	\end{subfigure}
	\hspace{0.2cm}
	\begin{subfigure}{0.49\textwidth} % M\'{a}s ancho para centrar
		\centering
		\includegraphics[width=\linewidth,page=1]{graphs/PDBs/1a9w/1a9w1ns.pdf}
		\caption{(4)}
	\end{subfigure}
	\caption{Regresiones lineales de $\log_{10}r$ vs $\log_{10}M(r)$ correspondiente a cuatro etapas de procesamiento de la quinta prote\'{i}na con \textit{IdPDB:1a9w} de la Tabla \ref{Tabla:ids9}: (1) Adici\'{o}n de \'{a}tomos de hidr\'{o}geno al sistema proteico; (2) al minimizar la energ\'{i´}a de la estructura molecular; (3) equilibrando el sistema bajo condiciones termodin\'{a}micas controladas; y (4) despu\'{e}s de una din\'{a}mica molecular de 1 ns.}
	\label{fig:1a9w}
\end{figure}



\begin{figure}[H]
	\subsection*{Prote\'{i}na Translocon EspA (IdPDB:7khw)}
	
	\hspace{-0.3cm} 
	\begin{subfigure}{0.49\textwidth}
		\centering
		\begin{overpic}[width=\linewidth]{graphs/PDBs/7khw/7khwaddH.pdf}
		\put(-8,34){\includegraphics[width=0.73\linewidth]{graphs/PDBs/7khw/7khwaddH.png}}
		\end{overpic}
		\caption{(1)}
	\end{subfigure}
	\hspace{0.2cm}
	\begin{subfigure}{0.49\textwidth}
		\centering
		\includegraphics[width=\linewidth,page=1]{graphs/PDBs/7khw/7khwEm.pdf}
		\caption{(2)}
	\end{subfigure}
	
	\vspace{0cm} % Espacio entre filas
	
	\hspace{-0.3cm} 
	\begin{subfigure}{0.49\textwidth}
		\centering
		\includegraphics[width=\linewidth,page=1]{graphs/PDBs/7khw/7khwEq.pdf}
		\caption{(3)}
	\end{subfigure}
	\hspace{0.2cm}
	\begin{subfigure}{0.49\textwidth} % M\'{a}s ancho para centrar
		\centering
		\includegraphics[width=\linewidth,page=1]{graphs/PDBs/7khw/7khw1ns.pdf}
		\caption{(4)}
	\end{subfigure}
	\caption{Regresiones lineales de $\log_{10}r$ vs $\log_{10}M(r)$ correspondiente a cuatro etapas de procesamiento de la novena prote\'{i}na con \textit{IdPDB:7khw} de la Tabla \ref{Tabla:ids9}: (1) Adici\'{o}n de \'{a}tomos de hidr\'{o}geno al sistema proteico; (2) al minimizar la energ\'{i´}a de la estructura molecular; (3) equilibrando el sistema bajo condiciones termodin\'{a}micas controladas; y (4) despu\'{e}s de una din\'{a}mica molecular de 1 ns.}
	\label{fig:7khw}
\end{figure}
